% !TeX document-id = {d8b4925c-2057-42a4-b894-2f1a3f1b6345}
%!TeX TXS-program:compile = txs:///xelatex/[--shell-escape]
\documentclass[aspectratio=169, mathserif]{beamer}	% TPU recommends 16:9 ratio, 4:3 may require some work with inner theme .sty file

% Style options:
% light --- light theme (default)
% dark --- dark theme
% enlogo --- english TPU logo {default}
% rulogo --- russian TPU logo

\usetheme[light, rulogo]{tpu}		% dark theme used as an example of optional argument

\usepackage[english, russian]{babel}		%uncomment this to work in russian
\usepackage[utf8]{inputenc}
\usepackage[T2A]{fontenc}

\usepackage{fontspec}

\setromanfont{Brygada1918}[
Path=./fonts/BrygadaFontFiles/,
Extension = .ttf,
UprightFont=*-Regular,
BoldFont=*-Bold,
ItalicFont=*-Italic,
BoldItalicFont=*-BoldItalic
]

\setsansfont{ALSSirius}[
Path=./fonts/ALSSiriusFiles/,
Extension = .otf,
UprightFont=*-Regular,
BoldFont=*-Bold,
%ItalicFont=*-Italic,
%BoldItalicFont=*-BoldItalic
]

\setmonofont{Consolas}[
Path=./fonts/ConsolasFontFiles/,
%Scale=0.85,
Extension = .ttf,
UprightFont=*-Regular,
BoldFont=*-Bold,
ItalicFont=*-Italic,
BoldItalicFont=*-BoldItalic
]

\usepackage[cache=false]{minted}
\usepackage{xcolor} % to access the named colour LightGray
\definecolor{LightGray}{gray}{0.9}
\definecolor{onedarkBckGr}{RGB}{40, 44, 52}

\usemintedstyle[python]{default}
\setminted[python]{
	fontsize=\scriptsize,
	escapeinside=||,
	mathescape=true,
	numbersep=5pt,
	gobble=2,
	linenos=true,
	frame=leftline,
	framesep=1mm,
	python3=true,
}

\usemintedstyle[pycon]{default}
\setminted[pycon]{
	fontsize=\scriptsize,
	escapeinside=||,
	mathescape=true,
	numbersep=5pt,
	gobble=2,
	linenos=false,
	frame=single,
	framesep=1mm,
	python3=true,
%	bgcolor=backcolour,
	linenos=true,
}

%\defaultfontfeatures{Ligatures={TeX},Renderer=Basic}  %% свойства шрифтов по умолчанию
%\setmainfont[Ligatures={TeX,Historic}]{Times New Roman} %% задаёт основной шрифт документа
%\setsansfont{Comic Sans MS}                    %% задаёт шрифт без засечек
%\setmonofont{Courier New}
%\usepackage[default]{droidserif}
%\usepackage[defaultsans]{droidsans}

\usepackage{booktabs}	% good looking tables
\usepackage{multicol}	% text in multiple columns, useful for side-by-side text and pictures
\usepackage{hyperref}
%\usepackage{minted}
\usepackage{xcolor}
\definecolor{maroon}{cmyk}{0, 0.87, 0.68, 0.32}
\definecolor{halfgray}{gray}{0.55}
\definecolor{ipython_frame}{RGB}{207, 207, 207}
\definecolor{ipython_bg}{RGB}{247, 247, 247}
\definecolor{ipython_red}{RGB}{186, 33, 33}
\definecolor{ipython_green}{RGB}{0, 128, 0}
\definecolor{ipython_cyan}{RGB}{64, 128, 128}
\definecolor{ipython_purple}{RGB}{170, 34, 255}
\definecolor{linkcolor}{HTML}{0000FF} % цвет гиперссылок
\definecolor{urlcolor}{HTML}{800080} % цвет ссылок
\definecolor{backcolour}{rgb}{0.95,0.95,0.92}

\usepackage{amsxtra}
\usepackage{longtable}
\usepackage{wrapfig}
\usepackage{ragged2e}
\usepackage[nooneline]{caption}
\DeclareCaptionTextFormat{center}{\centering{#1}}
\DeclareCaptionLabelFormat{figure}{Рисунок~#2}
\captionsetup[table]{justification=raggedleft,
	labelformat=empty,
	labelsep=endash,
	textformat=center,
	position=top,
	skip=5pt
}
\captionsetup[figure]{justification=centering,
	labelsep=endash,
	labelformat=figure,
	font={tiny}
}


\hyphenpenalty=10000	% i don’t think hyphenation in presentations is a good idea, feel free to change however you like

\includeonlyframes{c}

\title{\LARGE{Системный анализ процессов химической технологии}}
\subtitle{\textcolor{tpugreen}{\textbf{Лекция 8}} \\ \textbf{Моделирование химических реакций}}
\author[]{\textbf{Вячеслав Алексеевич Чузлов}}
\institute{к.т.н., доцент ОХИ ИШПР}
\date{\today}

\begin{document}

% notice usage of \titleframe and several other unconventional functions
% the reason being is custom backgrounds on these slides

\titleframe		% title

\tocframe{}		% this custom frame accepts options for ToC

%\addcontentsline{toc}{section}{\textbf{I Численные методы решения систем \\ линейных уравнений}}


\begin{frame}[fragile, ]{Численные методы решения систем \\ дифференциальных уравнений}
\scriptsize
Задачи, в которых необходимо решить систему из нескольких дифференциальных уравнений с несколькими искомыми функциями, очень распространены в предметной области химической технологии.
\vfill
Будем рассматривать системы, в которых число неизвестных функций совпадает с числом уравнений, разрешенных относительно производных.
\vfill
К примеру, система из двух уравнений с двумя неизвестными функциями $y$ и $z$ от одного и того же аргумента $x$ имеет вид:
\vfill
\begin{equation}\label{system-ODE}
	\left\{
	\begin{aligned}
		y' &= f_1\left(x, y, z\right)\\
		z' &= f_2\left(x, y, z\right)
	\end{aligned}
	\right.
\end{equation}
\vfill
\noindent при этом штрих означает производную по $x$.
\vfill
\end{frame}


\begin{frame}[fragile, ]{Численные методы решения систем \\ дифференциальных уравнений}
\scriptsize
Общий вид системы из $n$ уравнений с $n$ неизвестными функциями $x_1, x_2, \ldots, x_n$ от переменной $t$ имеет вид:
\vfill
\begin{equation}\label{system-ODE2}
	\left\{
	\begin{aligned}
		\dfrac{dx_1}{dt} &= f_1\left(t, x_1, x_2, \ldots, x_n\right) \\
		\dfrac{dx_2}{dt} &= f_2\left(t, x_1, x_2, \ldots, x_n\right) \\
		\ldots & \\
		\dfrac{dx_n}{dt} &= f_n\left(t, x_1, x_2, \ldots, x_n\right)
	\end{aligned}
	\right.
\end{equation}
\vfill
Ранее мы рассмотрели численные методы решения обыкновенных дифференциальных уравнений вида $y'=f(x,y)$ (методы Эйлера и Рунге-Кутты). Данные методы применяются и в случае решения систем обыкновенных дифференциальных уравнений.
\vfill
\end{frame}


\section{Метод Эйлера}
\sectionframe

\begin{frame}[fragile, ]{Метод Эйлера}
\scriptsize
Пусть дана следующая система обыкновенных дифференциальных уравнений:
\vfill
\begin{equation}
	\left\{
	\begin{aligned}
		\dfrac{dy_1}{dx} &= f_1 \left(x, y_1, y_2\right) \\
		\dfrac{dy_2}{dx} &= f_2\left(x, y_1, y_2\right)
	\end{aligned}
	\right.
\end{equation}
\vfill
\noindent с начальными условиями:
\vfill
\begin{equation}
	\begin{aligned}
		y_1 \big |_{x=x_0} &= y_{01} \\
		y_2 \big |_{x=x_0} &= y_{02}
	\end{aligned}
\end{equation}
\vfill
При использовании метода Эйлера, расчетные формулы примут следующий вид:
\vfill
\begin{equation}\label{Eiler_system}
	\left\{
	\begin{aligned}
		y_{(i), 1} &= y_{i-1,1} + h \cdot f_1 \left(x_{i-1}, y_{(i-1),1}, y_{(i-1),2}\right) \\
		y_{(i), 2} &= y_{i-1,2} + h \cdot f_2 \left(x_{i-1}, y_{(i-1),1}, y_{(i-1),2}\right) \\
		x_{i} &= x_{i-1} + h
	\end{aligned}
	\right.
\end{equation}
\vfill
\noindent где $h$~-- шаг интегрирования; $f_1\left(x_i, y_{i, 1}, y_{i, 2}\right)$ и $f_2\left(x_i, y_{i, 1}, y_{i, 2}\right)$~-- правые части дифференциальных уравнений.
\vfill
\end{frame}


\subsection{Пример~1}
\begin{frame}[fragile, ]{Пример~1}\label{slide:example1}
\scriptsize
Пусть требуется решить систему дифференциальных уравнений первого порядка:
\vfill
\begin{equation*}
	\left\{
	\begin{aligned}
		\dfrac{dy_1}{dx} &= y_2 \\
		\dfrac{dy_2}{dx} &= e^{-x\cdot y_1}
	\end{aligned}
	\right.
\end{equation*}
\vfill
\noindent методом Эйлера на отрезке $[0, 1]$ с шагом $h=0.1$.
\vfill
Начальные условия: $x_0 = 0$; $y_1(0)=0$; $y_2(0)=0$.
\vfill
Воспользуемся формулой~\eqref{Eiler_system} и запишем выражения для $y_{i,1}$ и $y_{i,2}$:
\vfill
\begin{equation*}
	\left\{
	\begin{aligned}
		y_{i,1} &= y_{(i-1),1} + 0.1 \cdot y_{(i-1),2} \\
		y_{i,2} &= y_{(i-1),2} + 0.1 \cdot e^{-x_{i-1}\cdot y_{(i-1),1}} \\
		x_i &= x_{i-1} + h
	\end{aligned}
	\right.
\end{equation*}
\vfill
\end{frame}

\begin{frame}[fragile, ]{Пример~1}
\scriptsize
Результаты вычислений сведем в таблице.
\vfill
\renewcommand{\arraystretch}{1.5}
\begin{longtable}{|r|r|r|r|}
%	\caption{}
%	\label{tab:eiler_system} \\

	\hline \multicolumn{1}{|r|}{$i$} & \multicolumn{1}{r|}{$x_i$} & \multicolumn{1}{r|}{$y_{i,1} = y_{(i-1),1} + 0.1 \cdot y_{(i-1),2}$} & \multicolumn{1}{r|}{$y_{i,2} = y_{(i-1),2} + 0.1 \cdot e^{-x_{i-1}\cdot y_{(i-1),1}}$}  \\
	\hline
	\endfirsthead

	\multicolumn{4}{r}{Продолжение таблицы \thetable{}} \\
	\hline
	\multicolumn{1}{|r|}{$i$} & \multicolumn{1}{r|}{$x_i$} & \multicolumn{1}{r|}{$y_{i,1} = y_{(i-1),1} + 0.1 \cdot y_{(i-1),2}$} & \multicolumn{1}{r|}{$y_{i,2} = y_{(i-1),2} + 0.1 \cdot e^{-x_{i-1}\cdot y_{(i-1),1}}$}  \\
	\hline
	\endhead

	$0$  &$0.0$ & $0.0000$ & $0.0000$  \\
	\hline

	$1$  &$0.1$ & $0.0000$ & $0.1000$  \\
	\hline

	$2$  &$0.2$ & $0.0100$ & $0.2000$  \\
	\hline

	$3$  &$0.3$ & $0.0300$ & $0.2998$ \\
	\hline

	$4$  &$0.4$ & $0.0600$ & $0.3989$  \\
	\hline

	$5$  &$0.5$ & $0.0999$ & $0.4965$  \\
	\hline

	$6$  &$0.6$ & $0.1495$ & $0.5917$  \\
	\hline

	$7$  &$0.7$ & $0.2087$ & $0.6831$  \\
	\hline

	$8$  &$0.8$ & $0.2770$ & $0.7695$  \\
	\hline

	$9$  &$0.9$ & $0.3539$ & $0.8496$  \\
	\hline

	$10$  &$1.0$ & $0.4389$ & $0.9223$  \\
	\hline
\end{longtable}
\vfill
\end{frame}

\subsubsection{Программная реализация}
\begin{frame}[fragile, ]{Программная реализация}
\scriptsize
\begin{minted}{python}
import numpy as np


def eiler(func, x0, xf, y0, h):
    count = int((xf - x0) / h) + 1
    y = [y0[:]]  # создание массива y с начальными условиями
    x = x0

    for i in range(1, count):
        right_parts = func(x, y[i-1])
        y.append([])  # добавление пустой строки

        for j in range(len(y0)):
            y[i].append(y[i-1][j] + h * right_parts[j])

        x += h

    return y
|\space|
|\space|
\end{minted}
\vfill
\end{frame}


\begin{frame}[fragile]{Программная реализация}
\scriptsize
\begin{minted}[firstnumber=last]{python}
def equations(x, y):  # Функция, содержащая правые части дифференциальных уравнений
    return [y[1], np.exp(-x * y[0])]


if __name__ == '__main__':
    print(eiler(equations, 0, 1, [0, 0], 0.1))
|\space|
\end{minted}
\vfill
\begin{minted}[frame=none, linenos=false]{pycon}
[[0, 0],
 [0.0, 0.1],
 [0.010000000000000002, 0.2],
 [0.030000000000000006, 0.29980019986673334],
 [0.05998001998667334, 0.39890423774402173],
 [0.09987044376107551, 0.4965335889709902],
 [0.14952380265817453, 0.5916626935059732],
 [0.20869007200877185, 0.6830819284599525],
 [0.2769982648547671, 0.7694905222116074],
 [0.35394731707592786, 0.8496142128887387],
 [0.4389087383648017, 0.9223342965713657]]
\end{minted}
\vfill
\end{frame}

\subsection{Пример~2}
\begin{frame}[fragile]{Пример~2}
\scriptsize
\alert{\textbf{Закон действующих масс}}
\vfill
Скорость химической реакции прямо пропорциональна произведению концентраций реагирующих веществ, возведенных в степени, равные их стехиометрическим коэффициентам.
\vfill
\begin{multicols}{2}
Схема химической реакции:
\vfill
$$
	n_1A_1 + n_2A_2 + n_3A_3 \overset{k}{\longrightarrow} B
$$
\vfill
Скорость данной реакции:
\vfill
$$
r = k \cdot \left[A_1\right]^{n_1} \cdot \left[A_2\right]^{n_2} \cdot \left[A_3\right]^{n_3}
$$
\vfill
\columnbreak
Изменение концентрации каждого компонента во времени:
\vfill
\begin{equation*}
	\left\{
	\begin{aligned}
		\dfrac{\partial C_{A_1}}{\partial t} &= -n_1 \cdot r \\
		\dfrac{\partial C_{A_2}}{\partial t} &= -n_2 \cdot r \\
		\dfrac{\partial C_{A_3}}{\partial t} &= -n_3 \cdot r \\
		\dfrac{\partial C_B}{\partial t} &= r
	\end{aligned}
	\right.
\end{equation*}
\end{multicols}
\vfill
где $k$~-- константа скорости химической реакции; $C_{A_1}$, $C_{A_2}$, $C_{A_3}$, $C_B$~-- концентрации веществ (моль/л), участвующих в химической реакции, $n_1$, $n_2$, $n_3$~-- стехиометрические коэффициенты в уравнении реакции.
\vfill
\end{frame}


\begin{frame}[fragile]{Пример~2}\label{slide:example2}
\scriptsize
Пусть дана схема химических реакций:
\vfill
$$
A \overset{k_1}{\underset{k_2}{\rightleftarrows}} B
$$
\vfill
Скорость прямой реакции: $r_1 = k_1 \cdot C_A$; скорость обратной реакции:
$r_2 = k_2 \cdot C_B$. Константы скоростей реакций: $k_1 = 0.85$; $k_2 = 0.1$, $C_A$ и $C_B$~-- концентрации компонентов $A$ и $B$.
Изменение концентрации реагирующих веществ во времени описывается следующей системой дифференциальных уравнений:
\vfill
\begin{equation*}
	\left\{
	\begin{aligned}
		\dfrac{\partial C_A}{\partial t} &= -r_1 + r_2 \\
		\dfrac{\partial C_B}{\partial t} &= r_1 - r_2
	\end{aligned}
	\right.
\end{equation*}
\vfill
Необходимо определить изменение концентрации каждого компонента по времени методом Эйлера на отрезке $[0, 1]$ с  шагом $h=0.1$. Начальные условия: $C_A(0) = 1$~(моль/л); $C_B(0) = 0$~(моль/л).
\vfill
Воспользуемся формулой~\eqref{Eiler_system} и запишем выражения для $C_{A,i}$ и $C_{B,i}$:
\vfill
\begin{equation*}
	\left\{
	\begin{aligned}
		C_{A, i} &= C_{A, (i-1)} + 0.1 \cdot \left(-k_1\cdot C_{A, (i-1)} + k_2 \cdot C_{B, (i-1)}\right) \\
		C_{B, i} &= C_{B, (i-1)} + 0.1 \cdot \left(k_1\cdot C_{A, (i-1)} - k_2 \cdot C_{B, (i-1)}\right) \\
		t_i &= t_{i-1} + h
	\end{aligned}
	\right.
\end{equation*}
\vfill
\end{frame}


\begin{frame}[fragile]{Пример 2}
\scriptsize
\begin{equation*}
	\left\{
	\begin{aligned}
		C_{A, i} &= C_{A, (i-1)} + 0.1 \cdot \left(-k_1\cdot C_{A, (i-1)} + k_2 \cdot C_{B, (i-1)}\right) \\
		C_{B, i} &= C_{B, (i-1)} + 0.1 \cdot \left(k_1\cdot C_{A, (i-1)} - k_2 \cdot C_{B, (i-1)}\right) \\
		t_i &= t_{i-1} + h
	\end{aligned}
	\right.
\end{equation*}
Результаты вычислений сведем в таблице.
\vfill
\begin{longtable}{|r|r|r|r|}
%	\caption{}
%	\label{tab:eiler_system2} \\

	\hline \multicolumn{1}{|r|}{$i$} & \multicolumn{1}{r|}{$t_i$} & \multicolumn{1}{r|}{$C_{A,i}$} & \multicolumn{1}{r|}{$C_{B,i}$}  \\
	\hline
	\endfirsthead

	\multicolumn{4}{r}{Продолжение таблицы \thetable{}} \\
	\hline
	\multicolumn{1}{|r|}{$i$} & \multicolumn{1}{r|}{$t_i$} & \multicolumn{1}{r|}{$C_{A,i}$} & \multicolumn{1}{r|}{$C_{B,i}$}  \\
	\hline
	\endhead

	$0$  &$0.0$ & $1.0000$ & $0.0000$  \\
	\hline

	$1$  &$0.1$ & $0.9150$ & $0.0850$  \\
	\hline

	$2$  &$0.2$ & $0.8381$ & $0.1619$  \\
	\hline

	$3$  &$0.3$ & $0.7685$ & $0.2315$ \\
	\hline

	$4$  &$0.4$ & $0.7055$ & $0.2945$  \\
	\hline

	$5$  &$0.5$ & $0.6484$ & $0.3516$  \\
	\hline

	$6$  &$0.6$ & $0.5968$ & $0.4032$  \\
	\hline

	$7$  &$0.7$ & $0.5501$ & $0.4499$  \\
	\hline

	$8$  &$0.8$ & $0.5079$ & $0.4921$  \\
	\hline

	$9$  &$0.9$ & $0.4696$ & $0.5304$  \\
	\hline

	$10$  &$1.0$ & $0.4350$ & $0.5650$  \\
	\hline
\end{longtable}
\vfill
\end{frame}


\subsubsection{Программная реализация}
\begin{frame}[fragile]{Программная реализация}
\scriptsize
\begin{minted}{python}
def equations(t, c, k):  # Функция, содержащая правые части дифф. уравнений
    right_parts = [
        -k[0] * c[0] + k[1] * c[1],
        k[0] * c[0] - k[1] * c[1],
    ]
    return right_parts


def eiler(func, x0, xf, y0, h, args=()):
    count = int((xf - x0) / h) + 1
    y, x = [y0[:]], x0
    for i in range(1, count):
        right_parts = func(x, y[i-1], *args)
        y.append([])
        for j in range(len(y0)):
            y[i].append(y[i-1][j] + h * right_parts[j])
        x += h
    return y
|\space|
|\space|
\end{minted}
\vfill
\end{frame}


\begin{frame}[fragile, label=c]{Программная реализация}
\scriptsize
\begin{minted}[firstnumber=last]{python}
k = [0.85, 0.1]
print(eiler(equations, 0, 1, [1, 0], 0.1, args=(k, )))
|\space|
\end{minted}
\vfill
\begin{minted}[frame=none, linenos=false]{pycon}
[[1, 0],
 [0.915, 0.085],
 [0.838075, 0.16192500000000004],
 [0.7684578750000001, 0.23154212500000004],
 [0.7054543768750001, 0.29454562312500004],
 [0.6484362110718751, 0.35156378892812506],
 [0.596834771020047, 0.4031652289799532],
 [0.5501354677731425, 0.4498645322268577],
 [0.5078725983346939, 0.4921274016653062],
 [0.469624701492898, 0.5303752985071022],
 [0.4350103548510727, 0.5649896451489275]]
\end{minted}
\vfill
\end{frame}


\section{Метод Рунге-Кутты}
\sectionframe


\begin{frame}[fragile, label=c]{Метод Рунге-Кутты}
\scriptsize
Пусть дана следующая система обыкновенных дифференциальных уравнений:
\vfill
\begin{equation}
	\left\{
	\begin{aligned}
		\dfrac{dy_1}{dx} &= f_1 \left(x, y_1, y_2\right) \\
		\dfrac{dy_2}{dx} &= f_2\left(x, y_1, y_2\right)
	\end{aligned}
	\right.
\end{equation}
\vfill
\noindent с начальными условиями:
\vfill
\begin{equation}
	\begin{aligned}
		y_1 \big |_{x=x_0} &= y_{01} \\
		y_2 \big |_{x=x_0} &= y_{02}
	\end{aligned}
\end{equation}
\vfill
\end{frame}


\begin{frame}[fragile, label=c]{Метод Рунге-Кутты}
\scriptsize
При использовании метода Рунге-Кутты, расчетные формулы примут следующий вид:
\vfill
\begin{equation}\label{RK_system}
	\left\{
	\begin{aligned}
		y_{i, 1} &= y_{(i-1),1} + h / 6 \cdot \left(k_{1,1} + 2 \cdot k_{2,1} + 2 \cdot k_{3,1} + k_{4,1}\right) \\
		y_{i, 2} &= y_{(i-1),2} + h / 6 \cdot \left(k_{1,2} + 2 \cdot k_{2,2} + 2 \cdot k_{3,2} + k_{4,2}\right) \\
		x_{i} &= x_{i-1} + h
	\end{aligned}
	\right.
\end{equation}
\vfill
\noindent где
\vfill
\begin{equation}\label{RK-params-system}
	\begin{tiny}
		\begin{aligned}
			k_{1,1} &= f_1\left(x, y_{(i-1),1}, y_{(i-1),2}\right); &\space
			k_{1,2} &= f_2\left(x, y_{(i-1),1}, y_{(i-1),2}\right); \\
			k_{2,1} &= f_1 \left(x + \dfrac{h}{2}, y_{(i-1),1} + k_{1,1}  \cdot \dfrac{h}{2}, y_{(i-1),2} + k_{1,2} \cdot \dfrac{h}{2}\right); &\space
			k_{2,2} &= f_2 \left(x + \dfrac{h}{2}, y_{(i-1),1} + k_{1,1} \cdot \dfrac{h}{2}, y_{(i-1),2} + k_{1,2} \cdot \dfrac{h}{2}\right); \\
			k_{3,1} &= f_1 \left(x + \dfrac{h}{2}, y_{(i-1),1} + k_{2,1}  \cdot \dfrac{h}{2}, y_{(i-1),2} + k_{2,2} \cdot \dfrac{h}{2}\right); &\space
			k_{3,2} &= f_2 \left(x + \dfrac{h}{2}, y_{(i-1),1} + k_{2,1} \cdot \dfrac{h}{2}, y_{(i-1),2} + k_{2,2} \cdot \dfrac{h}{2}\right); \\
			k_{4,1} &= f_1 \left(x + h, y_{(i-1),1} + k_{3,1}  \cdot h, y_{(i-1),2} + k_{3,2} \cdot h\right); &\space
			k_{4,2} &= f_2 \left(x + h, y_{(i-1),1} + k_{3,1} \cdot h, y_{(i-1),2} + k_{3,2} \cdot h \right).
		\end{aligned}
	\end{tiny}
\end{equation}
\vfill
\noindent где $h$~-- шаг интегрирования; $f_1\left(x_i, y_{(i-1), 1}, y_{(i-1), 2}\right)$ и $f_2\left(x_i, y_{(i-1), 1}, y_{(i-1), 2}\right)$~-- правые части дифференциальных уравнений, $k_{1,j}$, $k_{2,j}$, $k_{3,j}$, $k_{4,j}$~-- параметры метода Рунге-Кутты для $j$-го уравнения.
\vfill
\end{frame}


\subsection{Пример~1}
\begin{frame}[fragile, label=c]{Пример~1}
\scriptsize
Решим  методом Рунге-Кутты пример, приведенный на слайде~\ref{slide:example1}.
Воспользуемся формулами~\eqref{RK_system},~\eqref{RK-params-system} и запишем выражения для нахождения значений искомых переменных $y_{i,1}$ и $y_{i,2}$:
\vfill
\begin{equation*}
	\begin{aligned}
		k_{1,1} &= y_{(i-1),2}; &\qquad
		k_{1,2} &= \exp\left(-x_i\cdot y_{(i-1),1}\right); \\
		k_{2,1} &= y_{(i-1),2} + k_{1,2}\cdot \dfrac{h}{2}; &\qquad
		k_{2,2} &= \exp\left[-\left(x_i + \dfrac{h}{2}\right) \cdot \left(y_{(i-1),1} + k_{1,1} \cdot \dfrac{h}{2}\right)\right] \\
		k_{3,1} &= y_{(i-1),2} + k_{2,2}\cdot \dfrac{h}{2}; &\qquad
		k_{3,2} &= \exp\left[-\left(x_i + \dfrac{h}{2}\right) \cdot \left(y_{(i-1),1} + k_{2,1} \cdot \dfrac{h}{2}\right)\right] \\
		k_{4,1} &= y_{(i-1),2} + k_{3,2} \cdot h; &\qquad
		k_{4,2} &= \exp\left[-\left(x_i + h\right) \cdot \left(y_{(i-1),1} + k_{3,1} \cdot h\right)\right] \\
	\end{aligned}
\end{equation*}
\vfill
\begin{equation*}
	\left\{
	\begin{aligned}
		y_{i,1} &= y_{(i-1),1} + \dfrac{0.1}{6} \cdot \left(k_{1,1} + 2\cdot k_{2,1} + 2 \cdot k_{3,1} + k_{4,1}\right) \\
		y_{i,2} &= y_{(i-1),2} + \dfrac{0.1}{6} \cdot \left(k_{1,2} + 2\cdot k_{2,2} + 2 \cdot k_{3,2} + k_{4,2}\right) \\
		x_{i} &= x_{i-1} + 0.1
	\end{aligned}
	\right.
\end{equation*}
\vfill
\end{frame}


\begin{frame}[fragile, label=c]{Пример 1}
\scriptsize
Результаты вычислений сведем в таблице.
\vfill
\begin{longtable}{|r|r|r|r|r|r|r|r|r|r|r|r|}
%	\caption{}
%	\label{tab:RK_system} \\

	\hline
	\multicolumn{1}{|r|}{$i$} & \multicolumn{1}{r|}{$x_i$} & \multicolumn{1}{r|}{$k_{1,1}$} & \multicolumn{1}{r|}{$k_{2,1}$} & \multicolumn{1}{r|}{$k_{3,1}$} & \multicolumn{1}{r|}{$k_{4,1}$} & \multicolumn{1}{r|}{$y_{i,1}$} & \multicolumn{1}{r|}{$k_{1,2}$} & \multicolumn{1}{r|}{$k_{2,2}$} & \multicolumn{1}{r|}{$k_{3,2}$} & \multicolumn{1}{r|}{$k_{4,2}$} & \multicolumn{1}{r|}{$y_{i,2}$}  \\
	\hline
	\endfirsthead

	\multicolumn{12}{r}{Продолжение таблицы \thetable{}} \\
	\hline
	\multicolumn{1}{|r|}{$i$} & \multicolumn{1}{r|}{$x_i$} & \multicolumn{1}{r|}{$k_{1,1}$} & \multicolumn{1}{r|}{$k_{2,1}$} & \multicolumn{1}{r|}{$k_{3,1}$} & \multicolumn{1}{r|}{$k_{4,1}$} & \multicolumn{1}{r|}{$y_{i,1}$} & \multicolumn{1}{r|}{$k_{1,2}$} & \multicolumn{1}{r|}{$k_{2,2}$} & \multicolumn{1}{r|}{$k_{3,2}$} & \multicolumn{1}{r|}{$k_{4,2}$} & \multicolumn{1}{r|}{$y_{i,2}$}  \\
	\hline
	\endhead

	$0$ & $0.0$ & $-$ & $-$ & $-$ & $-$ & $0.0000$ & $-$ & $-$ & $-$ & $-$ & $0.0000$\\
	\hline
	$1$ & $0.1$ & $0.0000$ & $0.0500$ & $0.0500$ & $0.1000$ & $0.0050$ & $1.0000$ & $1.0000$ & $0.9999$ & $0.9995$ & $0.1000$\\
	\hline
	$2$ & $0.2$ & $0.1000$ & $0.1500$ & $0.1499$ & $0.1998$ & $0.0200$ & $0.9995$ & $0.9985$ & $0.9981$ & $0.9960$ & $0.1998$\\
	\hline
	$3$ & $0.3$ & $0.1998$ & $0.2496$ & $0.2494$ & $0.2990$ & $0.0449$ & $0.9960$ & $0.9925$ & $0.9919$ & $0.9866$ & $0.2990$\\
	\hline
	$4$ & $0.4$ & $0.2990$ & $0.3483$ & $0.3480$ & $0.3968$ & $0.0797$ & $0.9866$ & $0.9793$ & $0.9784$ & $0.9686$ & $0.3968$\\
	\hline
	$5$ & $0.5$ & $0.3968$ & $0.4453$ & $0.4446$ & $0.4923$ & $0.1242$ & $0.9686$ & $0.9562$ & $0.9551$ & $0.9398$ & $0.4924$\\
	\hline
	$6$ & $0.6$ & $0.4924$ & $0.5393$ & $0.5384$ & $0.5844$ & $0.1781$ & $0.9398$ & $0.9214$ & $0.9202$ & $0.8987$ & $0.5844$\\
	\hline
	$7$ & $0.7$ & $0.5844$ & $0.6293$ & $0.6281$ & $0.6716$ & $0.2409$ & $0.8987$ & $0.8739$ & $0.8727$ & $0.8448$ & $0.6717$\\
	\hline
	$8$ & $0.8$ & $0.6717$ & $0.7139$ & $0.7124$ & $0.7529$ & $0.3122$ & $0.8448$ & $0.8139$ & $0.8126$ & $0.7790$ & $0.7529$\\
	\hline
	$9$ & $0.9$ & $0.7529$ & $0.7919$ & $0.7901$ & $0.8271$ & $0.3913$ & $0.7790$ & $0.7427$ & $0.7415$ & $0.7032$ & $0.8271$\\
	\hline
	$10$ & $1.0$ & $0.8271$ & $0.8623$ & $0.8603$ & $0.8933$ & $0.4774$ & $0.7032$ & $0.6630$ & $0.6619$ & $0.6204$ & $0.8933$\\

	\hline
\end{longtable}
\vfill
\end{frame}


\subsubsection{Программная реализация}
\begin{frame}[fragile, label=c]{Программная реализация}
\scriptsize
\begin{minted}{python}
import numpy as np


def rk(func, x0, xf, y0, h):
    count = int((xf - x0) / h) + 1
    y = [y0[:]]
    x = x0
    for i in range(1, count):
        k1 = func(x, y[i-1])
        k2 = func(x + h / 2, [y + k * h / 2 for y, k in zip(y[i-1], k1)])
        k3 = func(x + h / 2, [y + k * h / 2 for y, k in zip(y[i-1], k2)])
        k4 = func(x + h, [y + k * h for y, k in zip(y[i-1], k3)])
        y.append([])
        for j in range(len(y0)):
            y[i].append(
                y[i-1][j] + h / 6 * (k1[j] + 2 * k2[j] + 2 * k3[j] + k4[j])
            )
        x += h
    return y
\end{minted}
\vfill
\end{frame}


\begin{frame}[fragile, label=c]{Программная реализация}
\scriptsize
\begin{minted}[firstnumber=last	]{python}
|\space|
|\space|
def equations(x, y):  # Функция, содержащая правые части дифф. уравнений
    return [y[1], np.exp(-x * y[0])]


if __name__ == '__main__':
    print(rk(equations, 0, 1, [0, 0], 0.1))
|\space|
\end{minted}
\vfill
\begin{minted}[frame=none, linenos=false]{pycon}
[[0, 0],
 [0.004999791679686959, 0.09998750234339197],
 [0.019992089353337197, 0.19980027824237273],
 [0.04493954532954178, 0.2989921821997826],
 [0.07974589273138522, 0.39683477618392093],
 [0.1242292261307227, 0.49235154280802335],
 [0.1781000081292174, 0.5843789596377397],
 [0.24094662432104696, 0.67165612248553],
 [0.3122311354618596, 0.7529375201538153],
 [0.39129695254854, 0.8271160064996047],
 [0.4773885589403407, 0.8933374434985747]]
\end{minted}
\vfill
\end{frame}


\subsection{Пример~2}
\begin{frame}[fragile, label=c]{Пример~2}
\scriptsize
Рассмотрим также решение примера, приведенного на слайде~\ref{slide:example2}, методом Рунге-Кутты.
Воспользуемся формулами~\eqref{RK_system},~\eqref{RK-params-system} и запишем выражения для нахождения значений искомых концентраций компонентов $C_{A,i}$ и $C_{B,i}$:
\vfill
\begin{equation*}
	\left\{
	\begin{aligned}
		C_{A,i} &= C_{A,(i-1)} + \dfrac{0.1}{6} \cdot \left(k_{1,1} + 2\cdot k_{2,1} + 2 \cdot k_{3,1} + k_{4,1}\right) \\
		C_{B,i} &= C_{B,(i-1)} + \dfrac{0.1}{6} \cdot \left(k_{1,2} + 2\cdot k_{2,2} + 2 \cdot k_{3,2} + k_{4,2}\right) \\
		t_{i} &= t_{i-1} + 0.1
	\end{aligned}
	\right.
\end{equation*}
\vfill
\begin{equation*}
	\begin{tiny}
		\begin{aligned}
			k_{1,1} &= -k_1  C_{A,(i-1)} + k_2  C_{B,(i-1)}; &
			k_{1,2} &= k_1  C_{A,(i-1)} - k_2  C_{B,(i-1)}; \\
			k_{2,1} &= -k_1  \left(C_{A,(i-1)} + k_{1,1}  \dfrac{h}{2}\right) + k_2  \left(C_{B,(i-1)}+ k_{1,2} \cdot \dfrac{h}{2}\right); &
			k_{2,2} &= k_1 \left(C_{A,(i-1)} + k_{1,1} \dfrac{h}{2}\right) - k_2 \left(C_{B,(i-1)}+ k_{1,2}  \dfrac{h}{2}\right); \\
			k_{3,1} &= -k_1  \left(C_{A,(i-1)} + k_{2,1}  \dfrac{h}{2}\right) + k_2  \left(C_{B,(i-1)}+ k_{2,2}  \dfrac{h}{2}\right); &
			k_{3,2} &= k_1  \left(C_{A,(i-1)} + k_{2,1}  \dfrac{h}{2}\right) - k_2  \left(C_{B,(i-1)}+ k_{2,2}  \dfrac{h}{2}\right); \\
			k_{4,1} &= -k_1  \left(C_{A,(i-1)} + k_{3,1}  h\right) + k_2  \left(C_{B,(i-1)}+ k_{3,2}  h\right); &
			k_{4,2} &= k_1  \left(C_{A,(i-1)} + k_{3,1}  h\right) - k_2  \left(C_{B,(i-1)}+ k_{3,2}  h\right) \\
		\end{aligned}
	\end{tiny}
\end{equation*}
\vfill
\end{frame}


\begin{frame}[fragile, label=c]{Пример~2}
\scriptsize
Результаты вычислений сведем в таблице.
\vfill
\AtBeginEnvironment{longtable}{\tiny}
\renewcommand{\arraystretch}{1.5}
\begin{longtable}{|r|r|r|r|r|r|r|r|r|r|r|r|}
%	\caption{}
%	\label{tab:RK_system2} \\
	\hline
	\multicolumn{1}{|r|}{$i$} & \multicolumn{1}{r|}{$t_i$} & \multicolumn{1}{r|}{$k_{1,1}$} & \multicolumn{1}{r|}{$k_{2,1}$} & \multicolumn{1}{r|}{$k_{3,1}$} & \multicolumn{1}{r|}{$k_{4,1}$} & \multicolumn{1}{r|}{$C_{A,i}$} & \multicolumn{1}{r|}{$k_{1,2}$} & \multicolumn{1}{r|}{$k_{2,2}$} & \multicolumn{1}{r|}{$k_{3,2}$} & \multicolumn{1}{r|}{$k_{4,2}$} & \multicolumn{1}{r|}{$C_{B,i}$}  \\
	\hline
	\endfirsthead

	\multicolumn{12}{r}{Продолжение таблицы \thetable{}} \\
	\hline
	\multicolumn{1}{|r|}{$i$} & \multicolumn{1}{r|}{$t_i$} & \multicolumn{1}{r|}{$k_{1,1}$} & \multicolumn{1}{r|}{$k_{2,1}$} & \multicolumn{1}{r|}{$k_{3,1}$} & \multicolumn{1}{r|}{$k_{4,1}$} & \multicolumn{1}{r|}{$C_{A,i}$} & \multicolumn{1}{r|}{$k_{1,2}$} & \multicolumn{1}{r|}{$k_{2,2}$} & \multicolumn{1}{r|}{$k_{3,2}$} & \multicolumn{1}{r|}{$k_{4,2}$} & \multicolumn{1}{r|}{$C_{B,i}$}  \\
	\hline
	\endhead

	$0$ & $0.0$ & $-$ & $-$ & $-$ & $-$ & $1.0000$ & $-$ & $-$ & $-$ & $-$ & $0.0000$\\
	\hline
	$1$ & $0.1$ & $-0.8500$ & $-0.8096$ & $-0.8115$ & $-0.7729$ & $0.9189$ & $0.8500$ & $0.8096$ & $0.8115$ & $0.7729$ & $0.0811$\\
	\hline
	$2$ & $0.2$ & $-0.7730$ & $-0.7363$ & $-0.7380$ & $-0.7029$ & $0.8452$ & $0.7730$ & $0.7363$ & $0.7380$ & $0.7029$ & $0.1548$\\
	\hline
	$3$ & $0.3$ & $-0.7029$ & $-0.6695$ & $-0.6711$ & $-0.6392$ & $0.7781$ & $0.7029$ & $0.6695$ & $0.6711$ & $0.6392$ & $0.2219$\\
	\hline
	$4$ & $0.4$ & $-0.6392$ & $-0.6088$ & $-0.6103$ & $-0.5812$ & $0.7171$ & $0.6392$ & $0.6088$ & $0.6103$ & $0.5812$ & $0.2829$\\
	\hline
	$5$ & $0.5$ & $-0.5813$ & $-0.5537$ & $-0.5550$ & $-0.5286$ & $0.6617$ & $0.5813$ & $0.5537$ & $0.5550$ & $0.5286$ & $0.3383$\\
	\hline
	$6$ & $0.6$ & $-0.5286$ & $-0.5035$ & $-0.5047$ & $-0.4807$ & $0.6113$ & $0.5286$ & $0.5035$ & $0.5047$ & $0.4807$ & $0.3887$\\
	\hline
	$7$ & $0.7$ & $-0.4807$ & $-0.4579$ & $-0.4589$ & $-0.4371$ & $0.5654$ & $0.4807$ & $0.4579$ & $0.4589$ & $0.4371$ & $0.4346$\\
	\hline
	$8$ & $0.8$ & $-0.4371$ & $-0.4164$ & $-0.4174$ & $-0.3975$ & $0.5237$ & $0.4371$ & $0.4164$ & $0.4174$ & $0.3975$ & $0.4763$\\
	\hline
	$9$ & $0.9$ & $-0.3975$ & $-0.3786$ & $-0.3795$ & $-0.3615$ & $0.4858$ & $0.3975$ & $0.3786$ & $0.3795$ & $0.3615$ & $0.5142$\\
	\hline
	$10$ & $1.0$ & $-0.3615$ & $-0.3443$ & $-0.3451$ & $-0.3287$ & $0.4513$ & $0.3615$ & $0.3443$ & $0.3451$ & $0.3287$ & $0.5487$\\
	\hline
\end{longtable}
\vfill
\end{frame}


\subsubsection{Программная реализация}
\begin{frame}[fragile, label=c]{Программная реализация}
\scriptsize
\begin{minted}{python}
def equations(t, c, k):  # Функция, содержащая правые части дифф. уравнений
    right_parts = [-k[0] * c[0] + k[1] * c[1],
                    k[0] * c[0] - k[1] * c[1]]
    return right_parts

def rk(func, x0, xf, y0, h, args=()):
    count = int((xf - x0) / h) + 1
    y = [y0[:]]
    x = x0
    for i in range(1, count):
        k1 = func(x, y[i-1], *args)
        k2 = func(x + h / 2, [y + k * h / 2 for y, k in zip(y[i-1], k1)], *args)
        k3 = func(x + h / 2, [y + k * h / 2 for y, k in zip(y[i-1], k2)], *args)
        k4 = func(x + h, [y + k * h for y, k in zip(y[i-1], k3)], *args)
        y.append([])
        for j in range(len(y0)):
            y[i].append(y[i-1][j] + h / 6 * (k1[j] + 2 * k2[j] + 2 * k3[j] + k4[j]))
        x += h
    return y
\end{minted}
\vfill
\end{frame}


\begin{frame}[fragile, label=c]{Программная реализация}
\scriptsize
\begin{minted}[firstnumber=last]{python}
|\space|
|\space|
if __name__ == '__main__':
    k = [0.85, 0.1]
    print(rk(equations, 0, 1, [1, 0], 0.1, args=(k, )))
|\space|
\end{minted}
\vfill
\begin{minted}[frame=none, linenos=false]{pycon}
[[1, 0],
 [0.9189126823697916, 0.08108731763020834],
 [0.8451740652412765, 0.15482593475872353],
 [0.7781181579189691, 0.22188184208103093],
 [0.717139326447514, 0.282860673552486],
 [0.6616868236612737, 0.33831317633872626],
 [0.6112598149591241, 0.388740185040876],
 [0.5654028548783672, 0.4345971451216329],
 [0.5237017736131901, 0.4762982263868099],
 [0.4857799363256236, 0.5142200636743764],
 [0.4512948414639336, 0.5487051585360664]]
\end{minted}
\vfill
\end{frame}


\begin{frame}[fragile, label=c]{Графическая визуализация}
\scriptsize
Построим графическую визуализацию полученного решения:
\vfill
\begin{figure}[h!]
	\centering
	\includegraphics[width=.6\linewidth]{./pics/figure_64}
	\caption{Изменение концентрации реагирующих веществ во времени}
	\label{fig:fig_64}
\end{figure}
\vfill
\end{frame}


\section{Расчет схемы химических реакций}
\begin{frame}[fragile, label=c]{Расчет схемы химических реакций}
\scriptsize
Рассмотрим следующую схему химических реакций:
\vfill
$$
	A \longleftrightarrow 2B \longleftrightarrow C
$$
\vfill
\noindent с константами скоростей $k_1$, $k_2$, $k_3$ и $k_4$.
Уравнения, описывающие скорость изменения концентраций компонентов по времени, записываются следующим образом:
\vfill
\begin{minipage}{.49\textwidth}
\begin{enumerate}
\item $A \longrightarrow 2B$ \qquad $r_1 = k_1 \cdot C_A$
\item $2B \longrightarrow A$ \qquad $r_2 = k_2 \cdot C_B ^2$
\item $2B \longrightarrow C$ \qquad $r_3 = k_3 \cdot C_B ^2$
\item $C \longrightarrow 2B$ \qquad $r_4 = k_4 \cdot C_C$
\end{enumerate}
\end{minipage}
\begin{minipage}{.5\textwidth}
\begin{equation*}
	\left\{
	\begin{aligned}
		\dfrac{dC_A}{dt} &= -r_1 + r_2 \\
		\dfrac{dC_B}{dt} &=  2 \cdot \left(r_1 - r_2 - r_3 + r_4\right) \\
		\dfrac{dC_C}{dt} &=  r_3 - r_4 \\
	\end{aligned}
	\right.
\end{equation*}
\end{minipage}
\vfill
\end{frame}


\subsection{Решение методом Эйлера}
\begin{frame}[fragile, label=c]{Метод Эйлера}
\scriptsize
\begin{minted}{python}
import numpy as np


def func(time: float, c: np.ndarray,
         k: np.ndarray) -> np.ndarray:
    ca, cb, cc = c
    k1, k2, k3, k4 = k
    r1, r2, r3, r4 = [
        k1 * ca,
        k2 * cb ** 2,
        k3 * cb ** 2,
        k4 * cc,
    ]
    dca_dt = -r1 + r2
    dcb_dt = 2 * (r1 - r2 - r3 + r4)
    dcc_dt = r3 - r4

    return dca_dt, dcb_dt, dcc_dt
|\space|
|\space|
\end{minted}
\vfill
\end{frame}


\begin{frame}[fragile, label=c]{Метод Эйлера}
\scriptsize
\begin{minted}[firstnumber=last]{python}
def eiler(func, x0, xf, y0, h, args=()):
    count = int((xf - x0) / h) + 1
    y = np.zeros((count, y0.shape[0]))
    x, y[0] = x0, y0[:].copy()
    for i in range(1, count):
        right_parts = func(x, y[i-1], *args)
        for j in range(len(y0)):
            y[i][j] = y[i-1][j] + h * right_parts[j]
        x += h
    return y


if __name__ == '__main__':
    k, y0 = np.array([.2, .1, .1, .1]), np.array([1, .5, .2])
    t0, tf = 0, 10
    y_eiler = eiler(func, t0, tf, y0, 0.1, args=(k, ))
|\space|
\end{minted}
\vfill
\end{frame}


\subsection{Решение методом Рунге-Кутты}
\begin{frame}[fragile, label=c]{Метод Рунге-Кутты}
\scriptsize
\begin{minted}{python}
import numpy as np


def func(time: float, c: np.ndarray,
         k: np.ndarray) -> np.ndarray:
    ca, cb, cc = c
    k1, k2, k3, k4 = k
    r1, r2, r3, r4 = [
        k1 * ca,
        k2 * cb ** 2,
        k3 * cb ** 2,
        k4 * cc,
    ]
    dca_dt = -r1 + r2
    dcb_dt = 2 * (r1 - r2 - r3 + r4)
    dcc_dt = r3 - r4

    return dca_dt, dcb_dt, dcc_dt
|\space|
|\space|
\end{minted}
\vfill
\end{frame}


\begin{frame}[fragile, label=c]{Метод Рунге-Кутты}
\scriptsize
\begin{minted}[firstnumber=last]{python}
def rk(func, x0, xf, y0, h, args=()):
    count = int((xf - x0) / h) + 1
    y = np.zeros((count, y0.shape[0]))
    x, y[0] = x0, y0[:].copy()
    for i in range(1, count):
        k1 = func(x, y[i-1], *args)
        k2 = func(x + h / 2, [y + k * h / 2 for y, k in zip(y[i-1], k1)], *args)
        k3 = func(x + h / 2, [y + k * h / 2 for y, k in zip(y[i-1], k2)], *args)
        k4 = func(x + h, [y + k * h for y, k in zip(y[i-1], k3)], *args)
        for j in range(len(y0)):
            y[i][j] = y[i-1][j] + h / 6 * (k1[j] + 2 * k2[j] + 2 * k3[j] + k4[j])
        x += h
    return y


if __name__ == '__main__':
    k, y0 = np.array([.2, .1, .1, .1]), np.array([1, .5, .2])
    t0, tf = 0, 10
    y_rk = rk(func, t0, tf, y0, 0.1, args=(k, ))
|\space|
\end{minted}
\vfill
\end{frame}



\contactsframe[\Large \textbf{Благодарю за внимание!}]{

	\bigskip
	\includegraphics[width=.05\textwidth]{pics/home} \quad Учебный корпус №2, ауд. 136 \\
	\includegraphics[width=.05\textwidth]{pics/mail} \quad chuva@tpu.ru \\
	\includegraphics[width=.03\textwidth]{pics/tel} \quad +7-962-782-66-15
}

\end{document}

