% !TeX document-id = {d8b4925c-2057-42a4-b894-2f1a3f1b6345}
%!TeX TXS-program:compile = txs:///xelatex/[--shell-escape]
\documentclass[aspectratio=169, mathserif]{beamer}	% TPU recommends 16:9 ratio, 4:3 may require some work with inner theme .sty file

% Style options:
% light --- light theme (default)
% dark --- dark theme
% enlogo --- english TPU logo {default}
% rulogo --- russian TPU logo

\usetheme[light, rulogo]{tpu}		% dark theme used as an example of optional argument

\usepackage[english, russian]{babel}		%uncomment this to work in russian
\usepackage[utf8]{inputenc}
\usepackage[T2A]{fontenc}

\usepackage{fontspec}

\setromanfont{Brygada1918}[
Path=./fonts/BrygadaFontFiles/,
Extension = .ttf,
UprightFont=*-Regular,
BoldFont=*-Bold,
ItalicFont=*-Italic,
BoldItalicFont=*-BoldItalic
]

\setsansfont{ALSSirius}[
Path=./fonts/ALSSiriusFiles/,
Extension = .otf,
UprightFont=*-Regular,
BoldFont=*-Bold,
%ItalicFont=*-Italic,
%BoldItalicFont=*-BoldItalic
]

\setmonofont{Consolas}[
Path=./fonts/ConsolasFontFiles/,
%Scale=0.85,
Extension = .ttf,
UprightFont=*-Regular,
BoldFont=*-Bold,
ItalicFont=*-Italic,
BoldItalicFont=*-BoldItalic
]

\usepackage[cache=false]{minted}
\usepackage{xcolor} % to access the named colour LightGray
\definecolor{LightGray}{gray}{0.9}
\definecolor{onedarkBckGr}{RGB}{40, 44, 52}

\usemintedstyle[python]{default}
\setminted[python]{
	fontsize=\scriptsize,
	escapeinside=||,
	mathescape=true,
	numbersep=5pt,
	gobble=2,
	linenos=true,
	frame=leftline,
	framesep=1mm,
	python3=true,
}

\usemintedstyle[pycon]{default}
\setminted[pycon]{
	fontsize=\scriptsize,
	escapeinside=||,
	mathescape=true,
	numbersep=5pt,
	gobble=2,
	linenos=false,
	frame=single,
	framesep=1mm,
	python3=true,
%	bgcolor=backcolour,
	linenos=true,
}

%\defaultfontfeatures{Ligatures={TeX},Renderer=Basic}  %% свойства шрифтов по умолчанию
%\setmainfont[Ligatures={TeX,Historic}]{Times New Roman} %% задаёт основной шрифт документа
%\setsansfont{Comic Sans MS}                    %% задаёт шрифт без засечек
%\setmonofont{Courier New}
%\usepackage[default]{droidserif}
%\usepackage[defaultsans]{droidsans}

\usepackage{booktabs}	% good looking tables
\usepackage{multicol}	% text in multiple columns, useful for side-by-side text and pictures
\usepackage{hyperref}
%\usepackage{minted}
\usepackage{xcolor}
\definecolor{maroon}{cmyk}{0, 0.87, 0.68, 0.32}
\definecolor{halfgray}{gray}{0.55}
\definecolor{ipython_frame}{RGB}{207, 207, 207}
\definecolor{ipython_bg}{RGB}{247, 247, 247}
\definecolor{ipython_red}{RGB}{186, 33, 33}
\definecolor{ipython_green}{RGB}{0, 128, 0}
\definecolor{ipython_cyan}{RGB}{64, 128, 128}
\definecolor{ipython_purple}{RGB}{170, 34, 255}
\definecolor{linkcolor}{HTML}{0000FF} % цвет гиперссылок
\definecolor{urlcolor}{HTML}{800080} % цвет ссылок
\definecolor{backcolour}{rgb}{0.95,0.95,0.92}

\usepackage{amsxtra}
\usepackage{longtable}
\usepackage{wrapfig}
\usepackage{ragged2e}
\usepackage[nooneline]{caption}
\DeclareCaptionTextFormat{center}{\centering{#1}}
\DeclareCaptionLabelFormat{figure}{Рисунок~#2}
\captionsetup[table]{justification=raggedleft,
	labelformat=empty,
	labelsep=endash,
	textformat=center,
	position=top,
	skip=5pt
}
\captionsetup[figure]{justification=centering,
	labelsep=endash,
	labelformat=figure,
	font={tiny}
}

\usepackage{listings}
\lstset{
	breaklines=true,
	%
	extendedchars=true,
	literate=
	{á}{{\'a}}1 {é}{{\'e}}1 {í}{{\'i}}1 {ó}{{\'o}}1 {ú}{{\'u}}1
	{Á}{{\'A}}1 {É}{{\'E}}1 {Í}{{\'I}}1 {Ó}{{\'O}}1 {Ú}{{\'U}}1
	{à}{{\`a}}1 {è}{{\`e}}1 {ì}{{\`i}}1 {ò}{{\`o}}1 {ù}{{\`u}}1
	{À}{{\`A}}1 {È}{{\'E}}1 {Ì}{{\`I}}1 {Ò}{{\`O}}1 {Ù}{{\`U}}1
	{ä}{{\"a}}1 {ë}{{\"e}}1 {ï}{{\"i}}1 {ö}{{\"o}}1 {ü}{{\"u}}1
	{Ä}{{\"A}}1 {Ë}{{\"E}}1 {Ï}{{\"I}}1 {Ö}{{\"O}}1 {Ü}{{\"U}}1
	{â}{{\^a}}1 {ê}{{\^e}}1 {î}{{\^i}}1 {ô}{{\^o}}1 {û}{{\^u}}1
	{Â}{{\^A}}1 {Ê}{{\^E}}1 {Î}{{\^I}}1 {Ô}{{\^O}}1 {Û}{{\^U}}1
	{œ}{{\oe}}1 {Œ}{{\OE}}1 {æ}{{\ae}}1 {Æ}{{\AE}}1 {ß}{{\ss}}1
	{ç}{{\c c}}1 {Ç}{{\c C}}1 {ø}{{\o}}1 {å}{{\r a}}1 {Å}{{\r A}}1
	{€}{{\EUR}}1 {£}{{\pounds}}1
}

%%
%% Python definition (c) 1998 Michael Weber
%% Additional definitions (2013) Alexis Dimitriadis
%% modified by me (should not have empty lines)
%%
\lstdefinelanguage{iPython}{
	morekeywords={access,and,break,class,continue,def,del,elif,else,except,exec,finally,for,from,global,if,import,in,is,lambda,not,or,pass,print,raise,return,try,while, nonlocal, yield, with},%
	%
	% Built-ins
	morekeywords=[2]{abs,all,any,basestring,bin,bool,bytearray,callable,chr,classmethod,cmp,compile,complex,delattr,dict,dir,divmod,enumerate,eval,execfile,file,filter,float,format,frozenset,getattr,globals,hasattr,hash,help,hex,id,input,int,isinstance,issubclass,iter,len,list,locals,long,map,max,memoryview,min,next,object,oct,open,ord,pow,property,range,raw_input,reduce,reload,repr,reversed,round,set,setattr,slice,sorted,staticmethod,str,sum,super,tuple,type,unichr,unicode,vars,xrange,zip,apply,buffer,coerce,intern, ascii, as, assert},%
	%
	sensitive=true,%
	morecomment=[l]\#,%
	morestring=[b]',%
	morestring=[b]",%
	%
	morestring=[s]{'''}{'''},% used for documentation text (mulitiline strings)
	morestring=[s]{"""}{"""},% added by Philipp Matthias Hahn
	%
	morestring=[s]{r'}{'},% `raw' strings
	morestring=[s]{r"}{"},%
	morestring=[s]{r'''}{'''},%
	morestring=[s]{r"""}{"""},%
	morestring=[s]{u'}{'},% unicode strings
	morestring=[s]{u"}{"},%
	morestring=[s]{u'''}{'''},%
	morestring=[s]{u"""}{"""},%
	morestring=[s]{b'}{'},% byte strings
	morestring=[s]{b"}{"},%
	morestring=[s]{b'''}{'''},%
	morestring=[s]{b"""}{"""},%
	%
	% {replace}{replacement}{lenght of replace}
	% *{-}{-}{1} will not replace in comments and so on
	literate=
	{á}{{\'a}}1 {é}{{\'e}}1 {í}{{\'i}}1 {ó}{{\'o}}1 {ú}{{\'u}}1
	{Á}{{\'A}}1 {É}{{\'E}}1 {Í}{{\'I}}1 {Ó}{{\'O}}1 {Ú}{{\'U}}1
	{à}{{\`a}}1 {è}{{\`e}}1 {ì}{{\`i}}1 {ò}{{\`o}}1 {ù}{{\`u}}1
	{À}{{\`A}}1 {È}{{\'E}}1 {Ì}{{\`I}}1 {Ò}{{\`O}}1 {Ù}{{\`U}}1
	{ä}{{\"a}}1 {ë}{{\"e}}1 {ï}{{\"i}}1 {ö}{{\"o}}1 {ü}{{\"u}}1
	{Ä}{{\"A}}1 {Ë}{{\"E}}1 {Ï}{{\"I}}1 {Ö}{{\"O}}1 {Ü}{{\"U}}1
	{â}{{\^a}}1 {ê}{{\^e}}1 {î}{{\^i}}1 {ô}{{\^o}}1 {û}{{\^u}}1
	{Â}{{\^A}}1 {Ê}{{\^E}}1 {Î}{{\^I}}1 {Ô}{{\^O}}1 {Û}{{\^U}}1
	{œ}{{\oe}}1 {Œ}{{\OE}}1 {æ}{{\ae}}1 {Æ}{{\AE}}1 {ß}{{\ss}}1
	{ç}{{\c c}}1 {Ç}{{\c C}}1 {ø}{{\o}}1 {å}{{\r a}}1 {Å}{{\r A}}1
	{€}{{\EUR}}1 {£}{{\pounds}}1,
	%
	literate=
	*{+}{{{\color{ipython_purple}+}}}1
	{-}{{{\color{ipython_purple}-}}}1
	{*}{{{\color{ipython_purple}$^\ast$}}}1
	{/}{{{\color{ipython_purple}/}}}1
	{^}{{{\color{ipython_purple}\^{}}}}1
	{?}{{{\color{ipython_purple}?}}}1
	{!}{{{\color{ipython_purple}!}}}1
	{\%}{{{\color{ipython_purple}\%}}}1
	{<}{{{\color{ipython_purple}<}}}1
	{>}{{{\color{ipython_purple}>}}}1
	{|}{{{\color{ipython_purple}|}}}1
	{\&}{{{\color{ipython_purple}\&}}}1
	{~}{{{\color{ipython_purple}~}}}1
	%
	%	{==}{{{\color{ipython_purple}==}}}2
	%	{<=}{{{\color{ipython_purple}<=}}}2
	%	{>=}{{{\color{ipython_purple}>=}}}2
	%
	%	{+=}{{{\color{ipython_purple}>=}}}2
	%	{&=}{{{\color{ipython_purple}>=}}}2
	%	{-=}{{{\color{ipython_purple}>=}}}2
	%	{|=}{{{\color{ipython_purple}>=}}}2
	%
	%	{*=}{{{\color{ipython_purple}>=}}}2
	%	{^=}{{{\color{ipython_purple}>=}}}2
	%	{/=}{{{\color{ipython_purple}>=}}}2
	%	{>>=}{{{\color{ipython_purple}>=}}}2
	%
	%	{\%=}{{{\color{ipython_purple}>=}}}2
	%	{<<=}{{{\color{ipython_purple}>=}}}2
	%	{**=}{{{\color{ipython_purple}>=}}}2
	%	{//=}{{{\color{ipython_purple}>=}}}2
	%
	{+=}{{{+=}}}2
	{-=}{{{-=}}}2
	{*=}{{{$^\ast$=}}}2
	{/=}{{{/=}}}2,
	%
	%	identifierstyle=\color{red}\ttfamily,
	commentstyle=\fontsize{7pt}{7}\color{ipython_cyan}\sffamily ,
	texcl=true,
	keepspaces=true,
	stringstyle=\fontsize{7pt}{7}\color{ipython_red}\ttfamily ,
	%	keepspaces=true,
	showspaces=false,
	showstringspaces=false,
	%
	rulecolor=\color{ipython_frame},
	frame=leftline,
	%	frameround=ffff,
	framexleftmargin=2mm,
	columns=fullflexible
	numbers=left,
	numberstyle=\tiny\color{halfgray},
	numbersep=14pt,
	%
	%
%		backgroundcolor=\color{ipython_bg},
	extendedchars=true,
	basicstyle=\fontsize{7pt}{7}\ttfamily,
	keywordstyle=\fontsize{7pt}{7}\color{ipython_green}\ttfamily,
	escapechar=\¢,escapebegin=\color{ipython_red},
}

\hyphenpenalty=10000	% i don’t think hyphenation in presentations is a good idea, feel free to change however you like

%\usepackage[breakable]{tcolorbox}
%    \usepackage{parskip} % Stop auto-indenting (to mimic markdown behaviour)

    \usepackage{iftex}
    \ifPDFTeX
        \usepackage[T1]{fontenc}
        \usepackage{mathpazo}
    \else
        \usepackage{fontspec}
    \fi

    % Basic figure setup, for now with no caption control since it's done
    % automatically by Pandoc (which extracts ![](path) syntax from Markdown).
%    \usepackage{graphicx}
    % Maintain compatibility with old templates. Remove in nbconvert 6.0
    \let\Oldincludegraphics\includegraphics
    % Ensure that by default, figures have no caption (until we provide a
    % proper Figure object with a Caption API and a way to capture that
    % in the conversion process - todo).
    \usepackage{caption}
    \DeclareCaptionFormat{nocaption}{}
    \captionsetup{format=nocaption,aboveskip=0pt,belowskip=0pt}

    \usepackage[Export]{adjustbox} % Used to constrain images to a maximum size
    \adjustboxset{max size={0.9\linewidth}{0.9\paperheight}}
    \usepackage{float}
    \floatplacement{figure}{H} % forces figures to be placed at the correct location
%    \usepackage{xcolor} % Allow colors to be defined
%    \usepackage{enumerate} % Needed for markdown enumerations to work
%    \usepackage{geometry} % Used to adjust the document margins
%    \usepackage{amsmath} % Equations
%    \usepackage{amssymb} % Equations
    \usepackage{textcomp} % defines textquotesingle
    % Hack from http://tex.stackexchange.com/a/47451/13684:
    \AtBeginDocument{%
        \def\PYZsq{\textquotesingle}% Upright quotes in Pygmentized code
    }
    \usepackage{upquote} % Upright quotes for verbatim code
    \usepackage{eurosym} % defines \euro
    \usepackage[mathletters]{ucs} % Extended unicode (utf-8) support
    \usepackage{fancyvrb} % verbatim replacement that allows latex
    \usepackage{grffile} % extends the file name processing of package graphics
                         % to support a larger range
    \makeatletter % fix for grffile with XeLaTeX
    \def\Gread@@xetex#1{%
      \IfFileExists{"\Gin@base".bb}%
      {\Gread@eps{\Gin@base.bb}}%
      {\Gread@@xetex@aux#1}%
    }
    \makeatother

    % The hyperref package gives us a pdf with properly built
    % internal navigation ('pdf bookmarks' for the table of contents,
    % internal cross-reference links, web links for URLs, etc.)
%    \usepackage{hyperref}
    % The default LaTeX title has an obnoxious amount of whitespace. By default,
    % titling removes some of it. It also provides customization options.
%    \usepackage{titling}
    \usepackage{longtable} % longtable support required by pandoc >1.10
    \usepackage{booktabs}  % table support for pandoc > 1.12.2
%    \usepackage[inline]{enumitem} % IRkernel/repr support (it uses the enumerate* environment)
    \usepackage[normalem]{ulem} % ulem is needed to support strikethroughs (\sout)
                                % normalem makes italics be italics, not underlines
    \usepackage{mathrsfs}



    % Colors for the hyperref package
    \definecolor{urlcolor}{rgb}{0,.145,.698}
    \definecolor{linkcolor}{rgb}{.71,0.21,0.01}
    \definecolor{citecolor}{rgb}{.12,.54,.11}

    % ANSI colors
    \definecolor{ansi-black}{HTML}{3E424D}
    \definecolor{ansi-black-intense}{HTML}{282C36}
    \definecolor{ansi-red}{HTML}{E75C58}
    \definecolor{ansi-red-intense}{HTML}{B22B31}
    \definecolor{ansi-green}{HTML}{00A250}
    \definecolor{ansi-green-intense}{HTML}{007427}
    \definecolor{ansi-yellow}{HTML}{DDB62B}
    \definecolor{ansi-yellow-intense}{HTML}{B27D12}
    \definecolor{ansi-blue}{HTML}{208FFB}
    \definecolor{ansi-blue-intense}{HTML}{0065CA}
    \definecolor{ansi-magenta}{HTML}{D160C4}
    \definecolor{ansi-magenta-intense}{HTML}{A03196}
    \definecolor{ansi-cyan}{HTML}{60C6C8}
    \definecolor{ansi-cyan-intense}{HTML}{258F8F}
    \definecolor{ansi-white}{HTML}{C5C1B4}
    \definecolor{ansi-white-intense}{HTML}{A1A6B2}
    \definecolor{ansi-default-inverse-fg}{HTML}{FFFFFF}
    \definecolor{ansi-default-inverse-bg}{HTML}{000000}

    % commands and environments needed by pandoc snippets
    % extracted from the output of `pandoc -s`
    \providecommand{\tightlist}{%
      \setlength{\itemsep}{0pt}\setlength{\parskip}{0pt}}
    \DefineVerbatimEnvironment{Highlighting}{Verbatim}{commandchars=\\\{\}}
    % Add ',fontsize=\small' for more characters per line
    \newenvironment{Shaded}{}{}
    \newcommand{\KeywordTok}[1]{\textcolor[rgb]{0.00,0.44,0.13}{\textbf{{#1}}}}
    \newcommand{\DataTypeTok}[1]{\textcolor[rgb]{0.56,0.13,0.00}{{#1}}}
    \newcommand{\DecValTok}[1]{\textcolor[rgb]{0.25,0.63,0.44}{{#1}}}
    \newcommand{\BaseNTok}[1]{\textcolor[rgb]{0.25,0.63,0.44}{{#1}}}
    \newcommand{\FloatTok}[1]{\textcolor[rgb]{0.25,0.63,0.44}{{#1}}}
    \newcommand{\CharTok}[1]{\textcolor[rgb]{0.25,0.44,0.63}{{#1}}}
    \newcommand{\StringTok}[1]{\textcolor[rgb]{0.25,0.44,0.63}{{#1}}}
    \newcommand{\CommentTok}[1]{\textcolor[rgb]{0.38,0.63,0.69}{\textit{{#1}}}}
    \newcommand{\OtherTok}[1]{\textcolor[rgb]{0.00,0.44,0.13}{{#1}}}
    \newcommand{\AlertTok}[1]{\textcolor[rgb]{1.00,0.00,0.00}{\textbf{{#1}}}}
    \newcommand{\FunctionTok}[1]{\textcolor[rgb]{0.02,0.16,0.49}{{#1}}}
    \newcommand{\RegionMarkerTok}[1]{{#1}}
    \newcommand{\ErrorTok}[1]{\textcolor[rgb]{1.00,0.00,0.00}{\textbf{{#1}}}}
    \newcommand{\NormalTok}[1]{{#1}}

    % Additional commands for more recent versions of Pandoc
    \newcommand{\ConstantTok}[1]{\textcolor[rgb]{0.53,0.00,0.00}{{#1}}}
    \newcommand{\SpecialCharTok}[1]{\textcolor[rgb]{0.25,0.44,0.63}{{#1}}}
    \newcommand{\VerbatimStringTok}[1]{\textcolor[rgb]{0.25,0.44,0.63}{{#1}}}
    \newcommand{\SpecialStringTok}[1]{\textcolor[rgb]{0.73,0.40,0.53}{{#1}}}
    \newcommand{\ImportTok}[1]{{#1}}
    \newcommand{\DocumentationTok}[1]{\textcolor[rgb]{0.73,0.13,0.13}{\textit{{#1}}}}
    \newcommand{\AnnotationTok}[1]{\textcolor[rgb]{0.38,0.63,0.69}{\textbf{\textit{{#1}}}}}
    \newcommand{\CommentVarTok}[1]{\textcolor[rgb]{0.38,0.63,0.69}{\textbf{\textit{{#1}}}}}
    \newcommand{\VariableTok}[1]{\textcolor[rgb]{0.10,0.09,0.49}{{#1}}}
    \newcommand{\ControlFlowTok}[1]{\textcolor[rgb]{0.00,0.44,0.13}{\textbf{{#1}}}}
    \newcommand{\OperatorTok}[1]{\textcolor[rgb]{0.40,0.40,0.40}{{#1}}}
    \newcommand{\BuiltInTok}[1]{{#1}}
    \newcommand{\ExtensionTok}[1]{{#1}}
    \newcommand{\PreprocessorTok}[1]{\textcolor[rgb]{0.74,0.48,0.00}{{#1}}}
    \newcommand{\AttributeTok}[1]{\textcolor[rgb]{0.49,0.56,0.16}{{#1}}}
    \newcommand{\InformationTok}[1]{\textcolor[rgb]{0.38,0.63,0.69}{\textbf{\textit{{#1}}}}}
    \newcommand{\WarningTok}[1]{\textcolor[rgb]{0.38,0.63,0.69}{\textbf{\textit{{#1}}}}}


    % Define a nice break command that doesn't care if a line doesn't already
    % exist.
    \def\br{\hspace*{\fill} \\* }
    % Math Jax compatibility definitions
    \def\gt{>}
    \def\lt{<}
    \let\Oldtex\TeX
    \let\Oldlatex\LaTeX
    \renewcommand{\TeX}{\textrm{\Oldtex}}
    \renewcommand{\LaTeX}{\textrm{\Oldlatex}}
    % Document parameters
    % Document title
    \title{Untitled8}

% Pygments definitions
\makeatletter
\def\PY@reset{\let\PY@it=\relax \let\PY@bf=\relax%
    \let\PY@ul=\relax \let\PY@tc=\relax%
    \let\PY@bc=\relax \let\PY@ff=\relax}
\def\PY@tok#1{\csname PY@tok@#1\endcsname}
\def\PY@toks#1+{\ifx\relax#1\empty\else%
    \PY@tok{#1}\expandafter\PY@toks\fi}
\def\PY@do#1{\PY@bc{\PY@tc{\PY@ul{%
    \PY@it{\PY@bf{\PY@ff{#1}}}}}}}
\def\PY#1#2{\PY@reset\PY@toks#1+\relax+\PY@do{#2}}

\expandafter\def\csname PY@tok@w\endcsname{\def\PY@tc##1{\textcolor[rgb]{0.73,0.73,0.73}{##1}}}
\expandafter\def\csname PY@tok@c\endcsname{\let\PY@it=\textit\def\PY@tc##1{\textcolor[rgb]{0.25,0.50,0.50}{##1}}}
\expandafter\def\csname PY@tok@cp\endcsname{\def\PY@tc##1{\textcolor[rgb]{0.74,0.48,0.00}{##1}}}
\expandafter\def\csname PY@tok@k\endcsname{\let\PY@bf=\textbf\def\PY@tc##1{\textcolor[rgb]{0.00,0.50,0.00}{##1}}}
\expandafter\def\csname PY@tok@kp\endcsname{\def\PY@tc##1{\textcolor[rgb]{0.00,0.50,0.00}{##1}}}
\expandafter\def\csname PY@tok@kt\endcsname{\def\PY@tc##1{\textcolor[rgb]{0.69,0.00,0.25}{##1}}}
\expandafter\def\csname PY@tok@o\endcsname{\def\PY@tc##1{\textcolor[rgb]{0.40,0.40,0.40}{##1}}}
\expandafter\def\csname PY@tok@ow\endcsname{\let\PY@bf=\textbf\def\PY@tc##1{\textcolor[rgb]{0.67,0.13,1.00}{##1}}}
\expandafter\def\csname PY@tok@nb\endcsname{\def\PY@tc##1{\textcolor[rgb]{0.00,0.50,0.00}{##1}}}
\expandafter\def\csname PY@tok@nf\endcsname{\def\PY@tc##1{\textcolor[rgb]{0.00,0.00,1.00}{##1}}}
\expandafter\def\csname PY@tok@nc\endcsname{\let\PY@bf=\textbf\def\PY@tc##1{\textcolor[rgb]{0.00,0.00,1.00}{##1}}}
\expandafter\def\csname PY@tok@nn\endcsname{\let\PY@bf=\textbf\def\PY@tc##1{\textcolor[rgb]{0.00,0.00,1.00}{##1}}}
\expandafter\def\csname PY@tok@ne\endcsname{\let\PY@bf=\textbf\def\PY@tc##1{\textcolor[rgb]{0.82,0.25,0.23}{##1}}}
\expandafter\def\csname PY@tok@nv\endcsname{\def\PY@tc##1{\textcolor[rgb]{0.10,0.09,0.49}{##1}}}
\expandafter\def\csname PY@tok@no\endcsname{\def\PY@tc##1{\textcolor[rgb]{0.53,0.00,0.00}{##1}}}
\expandafter\def\csname PY@tok@nl\endcsname{\def\PY@tc##1{\textcolor[rgb]{0.63,0.63,0.00}{##1}}}
\expandafter\def\csname PY@tok@ni\endcsname{\let\PY@bf=\textbf\def\PY@tc##1{\textcolor[rgb]{0.60,0.60,0.60}{##1}}}
\expandafter\def\csname PY@tok@na\endcsname{\def\PY@tc##1{\textcolor[rgb]{0.49,0.56,0.16}{##1}}}
\expandafter\def\csname PY@tok@nt\endcsname{\let\PY@bf=\textbf\def\PY@tc##1{\textcolor[rgb]{0.00,0.50,0.00}{##1}}}
\expandafter\def\csname PY@tok@nd\endcsname{\def\PY@tc##1{\textcolor[rgb]{0.67,0.13,1.00}{##1}}}
\expandafter\def\csname PY@tok@s\endcsname{\def\PY@tc##1{\textcolor[rgb]{0.73,0.13,0.13}{##1}}}
\expandafter\def\csname PY@tok@sd\endcsname{\let\PY@it=\textit\def\PY@tc##1{\textcolor[rgb]{0.73,0.13,0.13}{##1}}}
\expandafter\def\csname PY@tok@si\endcsname{\let\PY@bf=\textbf\def\PY@tc##1{\textcolor[rgb]{0.73,0.40,0.53}{##1}}}
\expandafter\def\csname PY@tok@se\endcsname{\let\PY@bf=\textbf\def\PY@tc##1{\textcolor[rgb]{0.73,0.40,0.13}{##1}}}
\expandafter\def\csname PY@tok@sr\endcsname{\def\PY@tc##1{\textcolor[rgb]{0.73,0.40,0.53}{##1}}}
\expandafter\def\csname PY@tok@ss\endcsname{\def\PY@tc##1{\textcolor[rgb]{0.10,0.09,0.49}{##1}}}
\expandafter\def\csname PY@tok@sx\endcsname{\def\PY@tc##1{\textcolor[rgb]{0.00,0.50,0.00}{##1}}}
\expandafter\def\csname PY@tok@m\endcsname{\def\PY@tc##1{\textcolor[rgb]{0.40,0.40,0.40}{##1}}}
\expandafter\def\csname PY@tok@gh\endcsname{\let\PY@bf=\textbf\def\PY@tc##1{\textcolor[rgb]{0.00,0.00,0.50}{##1}}}
\expandafter\def\csname PY@tok@gu\endcsname{\let\PY@bf=\textbf\def\PY@tc##1{\textcolor[rgb]{0.50,0.00,0.50}{##1}}}
\expandafter\def\csname PY@tok@gd\endcsname{\def\PY@tc##1{\textcolor[rgb]{0.63,0.00,0.00}{##1}}}
\expandafter\def\csname PY@tok@gi\endcsname{\def\PY@tc##1{\textcolor[rgb]{0.00,0.63,0.00}{##1}}}
\expandafter\def\csname PY@tok@gr\endcsname{\def\PY@tc##1{\textcolor[rgb]{1.00,0.00,0.00}{##1}}}
\expandafter\def\csname PY@tok@ge\endcsname{\let\PY@it=\textit}
\expandafter\def\csname PY@tok@gs\endcsname{\let\PY@bf=\textbf}
\expandafter\def\csname PY@tok@gp\endcsname{\let\PY@bf=\textbf\def\PY@tc##1{\textcolor[rgb]{0.00,0.00,0.50}{##1}}}
\expandafter\def\csname PY@tok@go\endcsname{\def\PY@tc##1{\textcolor[rgb]{0.53,0.53,0.53}{##1}}}
\expandafter\def\csname PY@tok@gt\endcsname{\def\PY@tc##1{\textcolor[rgb]{0.00,0.27,0.87}{##1}}}
\expandafter\def\csname PY@tok@err\endcsname{\def\PY@bc##1{\setlength{\fboxsep}{0pt}\fcolorbox[rgb]{1.00,0.00,0.00}{1,1,1}{\strut ##1}}}
\expandafter\def\csname PY@tok@kc\endcsname{\let\PY@bf=\textbf\def\PY@tc##1{\textcolor[rgb]{0.00,0.50,0.00}{##1}}}
\expandafter\def\csname PY@tok@kd\endcsname{\let\PY@bf=\textbf\def\PY@tc##1{\textcolor[rgb]{0.00,0.50,0.00}{##1}}}
\expandafter\def\csname PY@tok@kn\endcsname{\let\PY@bf=\textbf\def\PY@tc##1{\textcolor[rgb]{0.00,0.50,0.00}{##1}}}
\expandafter\def\csname PY@tok@kr\endcsname{\let\PY@bf=\textbf\def\PY@tc##1{\textcolor[rgb]{0.00,0.50,0.00}{##1}}}
\expandafter\def\csname PY@tok@bp\endcsname{\def\PY@tc##1{\textcolor[rgb]{0.00,0.50,0.00}{##1}}}
\expandafter\def\csname PY@tok@fm\endcsname{\def\PY@tc##1{\textcolor[rgb]{0.00,0.00,1.00}{##1}}}
\expandafter\def\csname PY@tok@vc\endcsname{\def\PY@tc##1{\textcolor[rgb]{0.10,0.09,0.49}{##1}}}
\expandafter\def\csname PY@tok@vg\endcsname{\def\PY@tc##1{\textcolor[rgb]{0.10,0.09,0.49}{##1}}}
\expandafter\def\csname PY@tok@vi\endcsname{\def\PY@tc##1{\textcolor[rgb]{0.10,0.09,0.49}{##1}}}
\expandafter\def\csname PY@tok@vm\endcsname{\def\PY@tc##1{\textcolor[rgb]{0.10,0.09,0.49}{##1}}}
\expandafter\def\csname PY@tok@sa\endcsname{\def\PY@tc##1{\textcolor[rgb]{0.73,0.13,0.13}{##1}}}
\expandafter\def\csname PY@tok@sb\endcsname{\def\PY@tc##1{\textcolor[rgb]{0.73,0.13,0.13}{##1}}}
\expandafter\def\csname PY@tok@sc\endcsname{\def\PY@tc##1{\textcolor[rgb]{0.73,0.13,0.13}{##1}}}
\expandafter\def\csname PY@tok@dl\endcsname{\def\PY@tc##1{\textcolor[rgb]{0.73,0.13,0.13}{##1}}}
\expandafter\def\csname PY@tok@s2\endcsname{\def\PY@tc##1{\textcolor[rgb]{0.73,0.13,0.13}{##1}}}
\expandafter\def\csname PY@tok@sh\endcsname{\def\PY@tc##1{\textcolor[rgb]{0.73,0.13,0.13}{##1}}}
\expandafter\def\csname PY@tok@s1\endcsname{\def\PY@tc##1{\textcolor[rgb]{0.73,0.13,0.13}{##1}}}
\expandafter\def\csname PY@tok@mb\endcsname{\def\PY@tc##1{\textcolor[rgb]{0.40,0.40,0.40}{##1}}}
\expandafter\def\csname PY@tok@mf\endcsname{\def\PY@tc##1{\textcolor[rgb]{0.40,0.40,0.40}{##1}}}
\expandafter\def\csname PY@tok@mh\endcsname{\def\PY@tc##1{\textcolor[rgb]{0.40,0.40,0.40}{##1}}}
\expandafter\def\csname PY@tok@mi\endcsname{\def\PY@tc##1{\textcolor[rgb]{0.40,0.40,0.40}{##1}}}
\expandafter\def\csname PY@tok@il\endcsname{\def\PY@tc##1{\textcolor[rgb]{0.40,0.40,0.40}{##1}}}
\expandafter\def\csname PY@tok@mo\endcsname{\def\PY@tc##1{\textcolor[rgb]{0.40,0.40,0.40}{##1}}}
\expandafter\def\csname PY@tok@ch\endcsname{\let\PY@it=\textit\def\PY@tc##1{\textcolor[rgb]{0.25,0.50,0.50}{##1}}}
\expandafter\def\csname PY@tok@cm\endcsname{\let\PY@it=\textit\def\PY@tc##1{\textcolor[rgb]{0.25,0.50,0.50}{##1}}}
\expandafter\def\csname PY@tok@cpf\endcsname{\let\PY@it=\textit\def\PY@tc##1{\textcolor[rgb]{0.25,0.50,0.50}{##1}}}
\expandafter\def\csname PY@tok@c1\endcsname{\let\PY@it=\textit\def\PY@tc##1{\textcolor[rgb]{0.25,0.50,0.50}{##1}}}
\expandafter\def\csname PY@tok@cs\endcsname{\let\PY@it=\textit\def\PY@tc##1{\textcolor[rgb]{0.25,0.50,0.50}{##1}}}

\def\PYZbs{\char`\\}
\def\PYZus{\char`\_}
\def\PYZob{\char`\{}
\def\PYZcb{\char`\}}
\def\PYZca{\char`\^}
\def\PYZam{\char`\&}
\def\PYZlt{\char`\<}
\def\PYZgt{\char`\>}
\def\PYZsh{\char`\#}
\def\PYZpc{\char`\%}
\def\PYZdl{\char`\$}
\def\PYZhy{\char`\-}
\def\PYZsq{\char`\'}
\def\PYZdq{\char`\"}
\def\PYZti{\char`\~}
% for compatibility with earlier versions
\def\PYZat{@}
\def\PYZlb{[}
\def\PYZrb{]}
\makeatother


    % For linebreaks inside Verbatim environment from package fancyvrb.
    \makeatletter
        \newbox\Wrappedcontinuationbox
        \newbox\Wrappedvisiblespacebox
        \newcommand*\Wrappedvisiblespace {\textcolor{red}{\textvisiblespace}}
        \newcommand*\Wrappedcontinuationsymbol {\textcolor{red}{\llap{\tiny$\m@th\hookrightarrow$}}}
        \newcommand*\Wrappedcontinuationindent {3ex }
        \newcommand*\Wrappedafterbreak {\kern\Wrappedcontinuationindent\copy\Wrappedcontinuationbox}
        % Take advantage of the already applied Pygments mark-up to insert
        % potential linebreaks for TeX processing.
        %        {, <, #, %, $, ' and ": go to next line.
        %        _, }, ^, &, >, - and ~: stay at end of broken line.
        % Use of \textquotesingle for straight quote.
        \newcommand*\Wrappedbreaksatspecials {%
            \def\PYGZus{\discretionary{\char`\_}{\Wrappedafterbreak}{\char`\_}}%
            \def\PYGZob{\discretionary{}{\Wrappedafterbreak\char`\{}{\char`\{}}%
            \def\PYGZcb{\discretionary{\char`\}}{\Wrappedafterbreak}{\char`\}}}%
            \def\PYGZca{\discretionary{\char`\^}{\Wrappedafterbreak}{\char`\^}}%
            \def\PYGZam{\discretionary{\char`\&}{\Wrappedafterbreak}{\char`\&}}%
            \def\PYGZlt{\discretionary{}{\Wrappedafterbreak\char`\<}{\char`\<}}%
            \def\PYGZgt{\discretionary{\char`\>}{\Wrappedafterbreak}{\char`\>}}%
            \def\PYGZsh{\discretionary{}{\Wrappedafterbreak\char`\#}{\char`\#}}%
            \def\PYGZpc{\discretionary{}{\Wrappedafterbreak\char`\%}{\char`\%}}%
            \def\PYGZdl{\discretionary{}{\Wrappedafterbreak\char`\$}{\char`\$}}%
            \def\PYGZhy{\discretionary{\char`\-}{\Wrappedafterbreak}{\char`\-}}%
            \def\PYGZsq{\discretionary{}{\Wrappedafterbreak\textquotesingle}{\textquotesingle}}%
            \def\PYGZdq{\discretionary{}{\Wrappedafterbreak\char`\"}{\char`\"}}%
            \def\PYGZti{\discretionary{\char`\~}{\Wrappedafterbreak}{\char`\~}}%
        }
        % Some characters . , ; ? ! / are not pygmentized.
        % This macro makes them "active" and they will insert potential linebreaks
        \newcommand*\Wrappedbreaksatpunct {%
            \lccode`\~`\.\lowercase{\def~}{\discretionary{\hbox{\char`\.}}{\Wrappedafterbreak}{\hbox{\char`\.}}}%
            \lccode`\~`\,\lowercase{\def~}{\discretionary{\hbox{\char`\,}}{\Wrappedafterbreak}{\hbox{\char`\,}}}%
            \lccode`\~`\;\lowercase{\def~}{\discretionary{\hbox{\char`\;}}{\Wrappedafterbreak}{\hbox{\char`\;}}}%
            \lccode`\~`\:\lowercase{\def~}{\discretionary{\hbox{\char`\:}}{\Wrappedafterbreak}{\hbox{\char`\:}}}%
            \lccode`\~`\?\lowercase{\def~}{\discretionary{\hbox{\char`\?}}{\Wrappedafterbreak}{\hbox{\char`\?}}}%
            \lccode`\~`\!\lowercase{\def~}{\discretionary{\hbox{\char`\!}}{\Wrappedafterbreak}{\hbox{\char`\!}}}%
            \lccode`\~`\/\lowercase{\def~}{\discretionary{\hbox{\char`\/}}{\Wrappedafterbreak}{\hbox{\char`\/}}}%
            \catcode`\.\active
            \catcode`\,\active
            \catcode`\;\active
            \catcode`\:\active
            \catcode`\?\active
            \catcode`\!\active
            \catcode`\/\active
            \lccode`\~`\~
        }
    \makeatother

    \let\OriginalVerbatim=\Verbatim
    \makeatletter
    \renewcommand{\Verbatim}[1][1]{%
        %\parskip\z@skip
        \sbox\Wrappedcontinuationbox {\Wrappedcontinuationsymbol}%
        \sbox\Wrappedvisiblespacebox {\FV@SetupFont\Wrappedvisiblespace}%
        \def\FancyVerbFormatLine ##1{\hsize\linewidth
            \vtop{\raggedright\hyphenpenalty\z@\exhyphenpenalty\z@
                \doublehyphendemerits\z@\finalhyphendemerits\z@
                \strut ##1\strut}%
        }%
        % If the linebreak is at a space, the latter will be displayed as visible
        % space at end of first line, and a continuation symbol starts next line.
        % Stretch/shrink are however usually zero for typewriter font.
        \def\FV@Space {%
            \nobreak\hskip\z@ plus\fontdimen3\font minus\fontdimen4\font
            \discretionary{\copy\Wrappedvisiblespacebox}{\Wrappedafterbreak}
            {\kern\fontdimen2\font}%
        }%

        % Allow breaks at special characters using \PYG... macros.
        \Wrappedbreaksatspecials
        % Breaks at punctuation characters . , ; ? ! and / need catcode=\active
        \OriginalVerbatim[#1,codes*=\Wrappedbreaksatpunct]%
    }
    \makeatother

    % Exact colors from NB
    \definecolor{incolor}{HTML}{303F9F}
    \definecolor{outcolor}{HTML}{D84315}
    \definecolor{cellborder}{HTML}{CFCFCF}
    \definecolor{cellbackground}{HTML}{F7F7F7}

    % prompt
    \makeatletter
    \newcommand{\boxspacing}{\kern\kvtcb@left@rule\kern\kvtcb@boxsep}
    \makeatother
    \newcommand{\prompt}[4]{
        \ttfamily\llap{{\color{#2}[#3]:\hspace{3pt}#4}}\vspace{-\baselineskip}
    }


\title{\LARGE{Системный анализ процессов химической технологии}}
\subtitle{\textcolor{tpugreen}{\textbf{Лекция 5}} \\ \textbf{Модули. Объектно-ориентированное \\ программирование}}
\author[]{\textbf{Вячеслав Алексеевич Чузлов}}
\institute{к.т.н., доцент ОХИ ИШПР}
\date{\today}

\begin{document}

% notice usage of \titleframe and several other unconventional functions
% the reason being is custom backgrounds on these slides

\titleframe		% title

\tocframe{}		% this custom frame accepts options for ToC

%\addcontentsline{toc}{section}{\textbf{I Численные методы решения систем \\ линейных уравнений}}

\section{Модули}
\sectionframe

\begin{frame}[fragile]{Модули}
\scriptsize
\begin{itemize}
	\item \textcolor{extraorange}{\textbf{Модуль}}~-- самая крупная организационная программная единица в Python, которая вмещает в себя программный код и данные, готовые для многократного использования.
	\item Каждый файл~-- это отдельный модуль, и модули могут импортировать другие модули для доступа к именам, которые в них определены.
	\item Обработка модулей выполняется двумя инструкциями:
	\vfill
	\mint{ipython}|import|
	\qquad Позволяет клиентам получать модуль целиком.
	\vfill
	\mint{ipython}|from|
	\qquad Позволяет клиентам получать определенные имена из модуля.
	\vfill
	\item Модули Python позволяют связывать индивидуальные файлы в более крупную программную систему.
\end{itemize}
\vfill
\end{frame}

\subsection{Назначение модулей}

\begin{frame}[fragile]{Назначение модулей}
\scriptsize
\textcolor{tpugreen}{\textbf{Многократное использование кода}}
\vfill
\begin{itemize}
	\item В отличие от кода, набираемого в интерактивном режиме Python, который исчезает после выхода из него, код в файлах модулей постоянен~-- его можно перезагружать и повторно запускать столько раз, сколько нужно.
	\vfill
	\item Модули представляют собой место для определения имен, известных как \textcolor{extraorange}{\textbf{атрибуты}}, на которые могут ссылаться многочисленные клиенты.
	\vfill
	\item Обеспечивается \textcolor{extraorange}{\textbf{модульная}} конструкция программ, группрующая функциональность в многократно используемые единицы.
\end{itemize}
\vfill
\end{frame}

\begin{frame}[fragile]{Назначение модулей}
\scriptsize
\textcolor{tpugreen}{\textbf{Разделение пространства имен системы}}
\vfill
\begin{itemize}
	\item Модули являются изолированными пакетами имен~-- нет возможности увидеть имя из другого файла, пока он не будет явно импортирован.
	\vfill
	\item Во многом подобно локальным областям видимости функций такое решение помогает избежать конфлоктов имен в программах.
	\vfill
	\item Модули являются интсрументами для гуппирования компонентов системы: прграммного кода и создаваемых объектов.
\end{itemize}
\vfill
\end{frame}

\begin{frame}[fragile]{Назначение модулей}
\scriptsize
\textcolor{tpugreen}{\textbf{Реализация разделяемых служб или данных}}
\vfill
\begin{itemize}
	\item Модули удобны для реализации компонентов, которые разделяются в рамках системы и потому требуют только одной копии.
	\vfill
	\item Например, если необходимо предоставить глобальный объект, применяемый в нескольких функциях или файлах, то его можно реализовать в модуле, который затем будет импортироваться многими клиентами.
\end{itemize}
\vfill
\end{frame}

\subsection{Создание модулей}
\begin{frame}[fragile]{Создание модулей}
\scriptsize
\begin{itemize}
\item Для определения модуля нужно набрать любой код Python и сохранить его в текстовом файле с расширением <<\texttt{.py}>>; любой подобный файл автоматически считается модулем Python.
\item Например, если поместить следующий оператор \mintinline{python}|def| в файл \texttt{module1.py} и импортировать его, то будет создан объект модуля с одним атрибутом~-- именем \texttt{printer}, которое будет ссылаться на объект функции:
\end{itemize}
\vfill
\begin{minted}{python}
# module1.py

def printer(x):
    print(x)
|\space|
\end{minted}
\vfill
\end{frame}

\section{Имена файлов модулей}
\begin{frame}[fragile]{Имена файлов модулей}
\scriptsize
\begin{itemize}
\item Имена файлов модулей должны заканчиваться суффиксом <<.py>>.
\item Имена модулей становятся именами переменных внутри программы Python (без суффикса <<.py>>), поэтому они также обязаны следовать обычным правилам именования переменных.
\item Например, можно создать файл модуля с именем \texttt{if.py}, но его нельзя будет импортировать. Так как \mintinline{python}|if| является зарезервированным словом, оператор \mintinline{python}|import if| приведет к синтаксической ошибке.
\end{itemize}
\vfill
\begin{minted}{pycon}
>>> import if
  File "<stdin>", line 1
    import if
           ^^
SyntaxError: invalid syntax
>>> |\space|
\end{minted}
\vfill
\end{frame}


\subsection{Инструкция \texttt{import}}
\begin{frame}[fragile]{Инструкция \texttt{import}}
\scriptsize
\begin{itemize}
	\item В операторе \mintinline{python}|import| просто указывается одно или несколько имен модулей для загрузки, разделенные запятыми.
	\item Так как оператор \mintinline{python}|import| дает имя, которое ссылается на полный объект модуля, мы обязаны задавать имя модуля, чтобы извлечь его атрибуты (например, \mintinline{python}|module1.printer|).
\end{itemize}
\vfill
\begin{minted}{pycon}
>>> import module1  # Модуль как единое целое (один или несколько)
>>> module1.printer('Hello world!')  # Указать имя модуля, чтобы получить имена
Hello world!
>>> |\space|
\end{minted}
\vfill
\end{frame}

\subsection{Инструкция \texttt{from ... import}}
\begin{frame}[fragile]{Инструкция \texttt{from ... import}}
\scriptsize
\begin{itemize}
	\item Оператор \mintinline{python}|from| дает возможность применять импортированные имена напрямую, не уточняя их именем модуля.
\vfill
\begin{minted}{pycon}
>>> from module1 import printer  # Импортировать объект (один или несколько)
>>> printer('Hello world!')      # Уточнение не требуется
Hello world!
>>> |\space|
\end{minted}
\vfill
\item Такая форма  \mintinline{python}|from| позволяет указывать одно или несколько имен для копирования, разделенных запятыми.
\end{itemize}
\vfill
\end{frame}

\subsection{Расширение \texttt{as} для операторов \texttt{import} и \texttt{from}}
\begin{frame}[fragile]{Расширение \texttt{as} для операторов \texttt{import} и \texttt{from}}
\scriptsize
\begin{itemize}
	\item Со временем операторы \mintinline{python}|import| и \mintinline{python}|from| были расширены, чтобы позволить назначать импортированному имени другое имя в сценарии:
\vfill
\begin{minted}{python}
import modulename as name  # Использовать name вместо modulename
|\space|
\end{minted}
\vfill
\item Это расширение часто применяется с целью предоставления кратких псевдонимов для более длинных имен и устранения конфликтов имен, когда в сценарии уже используется имя, которое иначе было бы перезаписано обычным оператором:
\vfill
\begin{minted}{python}
import reallylongname as name
from module1 import utility as util1
from module2 import utility as util2

name.func()
util1()
util2()
|\space|
\end{minted}
\item Если в новом выпуске библиотеки модуль или инструмент, широко используемый в вашем коде, получает новое имя, тогда Вы при импортировании всего лишь назначаете ему прежнее имя и предотвращаете нарушение работоспособности имеющегося кода:
\vfill
\begin{minted}{python}
import newname as oldname
from library import newname as oldname
|\space|
\end{minted}
\end{itemize}
\vfill
\end{frame}

\subsection{Смешанные режимы использования: \texttt{\_\_name\_\_} и \texttt{\_\_main\_\_}}
\begin{frame}[fragile]{Смешанные режимы использования:  \\ \texttt{\_\_name\_\_} и \texttt{\_\_main\_\_}}
\scriptsize
\begin{itemize}
	\item Каждый модуль имеет встроенный атрибут по имени \texttt{\_\_name\_\_}, который
	Python автоматически создает и присваивает следующим образом:
	\begin{enumerate}
		\scriptsize
		\item  Если файл запускается как программа верхнего уровня, тогда во время старта атрибут \texttt{\_\_name\_\_} устанавливается в строку \texttt{'\_\_main\_\_'}.

		\item  Если взамен файл импортируется, то атрибут \texttt{\_\_name\_\_} устанавливается в имя модуля, как известно его пользователям.
	\end{enumerate}
	\item Результатом будет то, что модуль может проверять собственный атрибут \texttt{\_\_name\_\_} для выяснения, запущен он или импортирован.
\end{itemize}
\vfill
Например, пусть создан следующий файл модуля по имени \texttt{runme.ру}, экспортирующий единственную функцию \texttt{tester}:
\vfill
\begin{minted}{python}
def tester():
    print("It's Christmas in Heaven...")

if __name__ == '__main__':  # Выполняется только когда запущен, а не импортирован
    print('Run as program')
    tester()
|\space|
\end{minted}
\vfill
\end{frame}

\begin{frame}[fragile]{Смешанные режимы использования:  \\ \texttt{\_\_name\_\_} и \texttt{\_\_main\_\_}}
\scriptsize
\begin{itemize}
	\item В модуле определена функция для импортирования и использования обычным образом:
\vfill
\begin{minted}{pycon}
>>> import runme
>>> runme.tester()
It's Christmas in Heaven...
>>> |\space|
\end{minted}
\vfill
\item Но модуль в самом конце содержит код, который настроен на автоматический вызов функции \texttt{tester}, когда данный файл запускается как программа:
\vfill
\begin{minted}{pycon}
c:\code> python runme.py
Run as program
It's Christmas in Heaven...
c:\code>
\end{minted}
\vfill
\item В сущности, переменная \texttt{\_\_name\_\_} модуля служит флагом режима использования, позволяющим его коду быть задействованным как импортируемая библиотека и как сценарий верхнего уровня.
\end{itemize}
\vfill
\end{frame}

\section{Объектно-ориентированное \\ программирование на языке Python}
\sectionframe

\begin{frame}[fragile]{Объектно-ориентированное программирование}
\scriptsize
\begin{itemize}
	\item \textcolor{tpugreen}{\textbf{Классы}} и \textcolor{tpugreen}{\textbf{объекты}}~-- это два основных аспекта объектно-ориентированного программирования. Класс создаёт новый тип, а объекты являются экземплярами класса.
%	\vfill
	\item Объекты могут хранить данные в обычных переменных, которые принадлежат объекту.
%	\vfill
	\item Переменные, принадлежащие объекту или классу, называют \textcolor{extraorange}{\textbf{полями}}.
%	\vfill
	\item Объекты могут также обладать функционалом, т.е. иметь функции, принадлежащие классу. Такие функции принято называть \textcolor{extraorange}{\textbf{методами}} класса.
%	\vfill
	\item Всё вместе (поля и методы) принято называть \textcolor{extraorange}{\textbf{атрибутами}} класса.
\end{itemize}
\vfill
\end{frame}

\subsection{Классы и объекты}
\begin{frame}[fragile]{Классы и объекты}
\scriptsize
\begin{itemize}
	\item \textcolor{tpugreen}{\textbf{Класс}}~-- это способ описания сущности, определяющий состояние и поведение, зависящее от этого состояния, а также правила для взаимодействия с данной сущностью (контракт).
\vfill
    \begin{tcolorbox}[breakable, size=fbox, boxrule=1pt, pad at break*=1mm,colback=cellbackground, colframe=cellborder]
\prompt{In}{incolor}{1}{\boxspacing}
\begin{Verbatim}[commandchars=\\\{\}]
\PY{k}{class} \PY{n+nc}{Person}\PY{p}{:}
    \PY{k}{def} \PY{n+nf}{go}\PY{p}{(}\PY{n+nb+bp}{self}\PY{p}{,} \PY{n}{where}\PY{o}{=}\PY{l+s+s1}{\PYZsq{}}\PY{l+s+s1}{nowhere}\PY{l+s+s1}{\PYZsq{}}\PY{p}{)}\PY{p}{:}
        \PY{n+nb}{print}\PY{p}{(}\PY{n}{where}\PY{p}{)}
\end{Verbatim}
\end{tcolorbox}
\vfill
	\item \textcolor{tpugreen}{\textbf{Объект}} (\textbf{экземпляр})~-- это отдельный представитель класса, имеющий конкретное состояние и поведение, полностью определяемое классом.
	Объект имеет конкретные значения атрибутов и методы, работающие с этими значениями на основе правил, заданных в классе.
\vfill
    \begin{tcolorbox}[breakable, size=fbox, boxrule=1pt, pad at break*=1mm,colback=cellbackground, colframe=cellborder]
\prompt{In}{incolor}{2}{\boxspacing}
\begin{Verbatim}[commandchars=\\\{\}]
\PY{n}{petr} \PY{o}{=} \PY{n}{Person}\PY{p}{(}\PY{p}{)}
\PY{n}{ivan} \PY{o}{=} \PY{n}{Person}\PY{p}{(}\PY{p}{)}

\PY{n+nb}{print}\PY{p}{(}\PY{n}{petr}\PY{p}{)}
\end{Verbatim}
\end{tcolorbox}

    \begin{Verbatim}[commandchars=\\\{\}]
<\_\_main\_\_.Person object at 0x000001C97D98DD90>
    \end{Verbatim}
\vfill
\item Методы класса должны иметь дополнительно имя, добавляемое к началу списка параметров. Однако, при вызове метода никакого значения этому параметру присваивать не нужно~-- его укажет Python. Эта переменная указывает на сам объект экземпляра класса, и по традиции она называется \mintinline{python}|self|.
\end{itemize}
\vfill
\end{frame}

\subsection{Конструктор}
\begin{frame}[fragile]{Конструктор}
\scriptsize
\textbf{Конструктор класса}~-- специальный блок инструкций, вызываемый при создании объекта. В Python это \mintinline{python}|__init__| метод.
\vfill
    \begin{tcolorbox}[breakable, size=fbox, boxrule=1pt, pad at break*=1mm,colback=cellbackground, colframe=cellborder]
\prompt{In}{incolor}{1}{\boxspacing}
\begin{Verbatim}[commandchars=\\\{\}]
\PY{k}{class} \PY{n+nc}{Person}\PY{p}{(}\PY{n+nb}{object}\PY{p}{)}\PY{p}{:}
    \PY{k}{def} \PY{n+nf+fm}{\PYZus{}\PYZus{}init\PYZus{}\PYZus{}}\PY{p}{(}\PY{n+nb+bp}{self}\PY{p}{,} \PY{n}{name}\PY{p}{,} \PY{n}{surname}\PY{p}{)}\PY{p}{:}
        \PY{n+nb+bp}{self}\PY{o}{.}\PY{n}{name} \PY{o}{=} \PY{n}{name}
        \PY{n+nb+bp}{self}\PY{o}{.}\PY{n}{surname} \PY{o}{=} \PY{n}{surname}

\PY{n}{person} \PY{o}{=} \PY{n}{Person}\PY{p}{(}\PY{n}{name}\PY{o}{=}\PY{l+s+s1}{\PYZsq{}}\PY{l+s+s1}{Petr}\PY{l+s+s1}{\PYZsq{}}\PY{p}{,} \PY{n}{surname}\PY{o}{=}\PY{l+s+s1}{\PYZsq{}}\PY{l+s+s1}{Petrov}\PY{l+s+s1}{\PYZsq{}}\PY{p}{)}

\PY{n+nb}{print}\PY{p}{(}\PY{l+s+sa}{f}\PY{l+s+s1}{\PYZsq{}}\PY{l+s+s1}{Hi, }\PY{l+s+si}{\PYZob{}}\PY{n}{person}\PY{o}{.}\PY{n}{name}\PY{l+s+si}{\PYZcb{}}\PY{l+s+s1}{ }\PY{l+s+si}{\PYZob{}}\PY{n}{person}\PY{o}{.}\PY{n}{surname}\PY{l+s+si}{\PYZcb{}}\PY{l+s+s1}{!}\PY{l+s+s1}{\PYZsq{}}\PY{p}{)}

\PY{k}{for} \PY{n}{person} \PY{o+ow}{in} \PY{p}{[}\PY{n}{Person}\PY{p}{(}\PY{l+s+s1}{\PYZsq{}}\PY{l+s+s1}{Maxim}\PY{l+s+s1}{\PYZsq{}}\PY{p}{,} \PY{l+s+s1}{\PYZsq{}}\PY{l+s+s1}{Ivanov}\PY{l+s+s1}{\PYZsq{}}\PY{p}{)}\PY{p}{,} \PY{n}{Person}\PY{p}{(}\PY{l+s+s1}{\PYZsq{}}\PY{l+s+s1}{Mariya}\PY{l+s+s1}{\PYZsq{}}\PY{p}{,} \PY{l+s+s1}{\PYZsq{}}\PY{l+s+s1}{Petrova}\PY{l+s+s1}{\PYZsq{}}\PY{p}{)}\PY{p}{]}\PY{p}{:}
    \PY{n+nb}{print}\PY{p}{(}\PY{l+s+sa}{f}\PY{l+s+s1}{\PYZsq{}}\PY{l+s+s1}{Hi, }\PY{l+s+si}{\PYZob{}}\PY{n}{person}\PY{o}{.}\PY{n}{name}\PY{l+s+si}{\PYZcb{}}\PY{l+s+s1}{ }\PY{l+s+si}{\PYZob{}}\PY{n}{person}\PY{o}{.}\PY{n}{surname}\PY{l+s+si}{\PYZcb{}}\PY{l+s+s1}{!}\PY{l+s+s1}{\PYZsq{}}\PY{p}{)}
\end{Verbatim}
\end{tcolorbox}

    \begin{Verbatim}[commandchars=\\\{\}]
Hi, Petr Petrov!
Hi, Maxim Ivanov!
Hi, Mariya Petrova!
    \end{Verbatim}
\vfill
\end{frame}

\subsection{Переменные класса и объекта}
\begin{frame}[fragile]{Переменные класса и объекта}
\scriptsize
\begin{itemize}
	\item Поля можно воспринимать как обычные переменные, заключённые в \textbf{пространствах имён} классов и объектов. Их имена действительны только в контексте (пространстве имен) этих классов или объектов.

	\item \textcolor{extraorange}{\textbf{Переменные класса}} разделяемы~-- доступ к ним могут получать все экземпляры этого класса. Переменная класса существует только одна, поэтому когда любой из объектов изменяет переменную класса, это изменение отразится и во всех остальных экземплярах класса.

	\item \textcolor{extraorange}{\textbf{Переменные объекта}} принадлежат каждому отдельному экземпляру класса. В этом случае у каждого объекта есть своя собственная копия поля, т.е. не разделяемая с другими такими же полями в других экземплярах. Доступ к полям объекта осуществляется через переменную \mintinline{python}|self|.
\end{itemize}
\vfill
\end{frame}

\begin{frame}[fragile]{Переменные класса и объекта}
\scriptsize
    \begin{tcolorbox}[breakable, size=fbox, boxrule=1pt, pad at break*=1mm,colback=cellbackground, colframe=cellborder]
\prompt{In}{incolor}{1}{\boxspacing}
\begin{Verbatim}[commandchars=\\\{\}]
\PY{k}{class} \PY{n+nc}{Droid}\PY{p}{:}
    \PY{n}{population} \PY{o}{=} \PY{l+m+mi}{0}

    \PY{k}{def} \PY{n+nf+fm}{\PYZus{}\PYZus{}init\PYZus{}\PYZus{}}\PY{p}{(}\PY{n+nb+bp}{self}\PY{p}{,} \PY{n}{name}\PY{p}{)}\PY{p}{:}
        \PY{n+nb+bp}{self}\PY{o}{.}\PY{n}{name} \PY{o}{=} \PY{n}{name}
        \PY{n+nb}{print}\PY{p}{(}\PY{l+s+sa}{f}\PY{l+s+s1}{\PYZsq{}}\PY{l+s+s1}{  **Инициализация }\PY{l+s+si}{\PYZob{}}\PY{n+nb+bp}{self}\PY{o}{.}\PY{n}{name}\PY{l+s+si}{\PYZcb{}}\PY{l+s+s1}{**}\PY{l+s+s1}{\PYZsq{}}\PY{p}{)}
        \PY{n}{Droid}\PY{o}{.}\PY{n}{population} \PY{o}{+}\PY{o}{=} \PY{l+m+mi}{1}

    \PY{k}{def} \PY{n+nf}{say\PYZus{}hi}\PY{p}{(}\PY{n+nb+bp}{self}\PY{p}{)}\PY{p}{:}
        \PY{n+nb}{print}\PY{p}{(}\PY{l+s+sa}{f}\PY{l+s+s1}{\PYZsq{}}\PY{l+s+s1}{Приветствую! Мои хозяева называют меня }\PY{l+s+si}{\PYZob{}}\PY{n+nb+bp}{self}\PY{o}{.}\PY{n}{name}\PY{l+s+si}{\PYZcb{}}\PY{l+s+s1}{.}\PY{l+s+s1}{\PYZsq{}}\PY{p}{)}

    \PY{n+nd}{@staticmethod}
    \PY{k}{def} \PY{n+nf}{how\PYZus{}many}\PY{p}{(}\PY{p}{)}\PY{p}{:}
        \PY{n+nb}{print}\PY{p}{(}\PY{l+s+sa}{f}\PY{l+s+s1}{\PYZsq{}}\PY{l+s+s1}{У нас }\PY{l+s+si}{\PYZob{}}\PY{n}{Droid}\PY{o}{.}\PY{n}{population}\PY{l+s+si}{\PYZcb{}}\PY{l+s+s1}{ дроидов.}\PY{l+s+s1}{\PYZsq{}}\PY{p}{)}
\end{Verbatim}
\end{tcolorbox}
\vfill
\end{frame}

\begin{frame}[fragile]{Переменные класса и объекта}
\scriptsize
    \begin{tcolorbox}[breakable, size=fbox, boxrule=1pt, pad at break*=1mm,colback=cellbackground, colframe=cellborder]
\prompt{In}{incolor}{2}{\boxspacing}
\begin{Verbatim}[commandchars=\\\{\}]
\PY{n}{droid1} \PY{o}{=} \PY{n}{Droid}\PY{p}{(}\PY{l+s+s1}{\PYZsq{}}\PY{l+s+s1}{R2\PYZhy{}D2}\PY{l+s+s1}{\PYZsq{}}\PY{p}{)}
\PY{n}{droid1}\PY{o}{.}\PY{n}{say\PYZus{}hi}\PY{p}{(}\PY{p}{)}

\PY{n}{Droid}\PY{o}{.}\PY{n}{how\PYZus{}many}\PY{p}{(}\PY{p}{)}

\PY{n}{droid2} \PY{o}{=} \PY{n}{Droid}\PY{p}{(}\PY{l+s+s1}{\PYZsq{}}\PY{l+s+s1}{C\PYZhy{}3PO}\PY{l+s+s1}{\PYZsq{}}\PY{p}{)}
\PY{n}{droid2}\PY{o}{.}\PY{n}{say\PYZus{}hi}\PY{p}{(}\PY{p}{)}
\PY{n}{Droid}\PY{o}{.}\PY{n}{how\PYZus{}many}\PY{p}{(}\PY{p}{)}
\end{Verbatim}
\end{tcolorbox}

    \begin{Verbatim}[commandchars=\\\{\}]
  **Инициализация R2-D2**
Приветствую! Мои хозяева называют меня R2-D2.
У нас 1 дроидов.
  **Инициализация C-3PO**
Приветствую! Мои хозяева называют меня C-3PO.
У нас 2 дроидов.
    \end{Verbatim}
\end{frame}

\begin{frame}[fragile]{Переменные класса и объекта}
\scriptsize
    \begin{tcolorbox}[breakable, size=fbox, boxrule=1pt, pad at break*=1mm,colback=cellbackground, colframe=cellborder]
\prompt{In}{incolor}{3}{\boxspacing}
\begin{Verbatim}[commandchars=\\\{\}]
\PY{k}{def} \PY{n+nf}{how\PYZus{}many}\PY{p}{(}\PY{n}{obj}\PY{p}{:} \PY{n}{Droid}\PY{p}{)} \PY{o}{\PYZhy{}}\PY{o}{\PYZgt{}} \PY{n+nb}{int}\PY{p}{:}
    \PY{k}{return} \PY{n}{obj}\PY{o}{.}\PY{n}{population}

\PY{n}{droids\PYZus{}count} \PY{o}{=} \PY{n}{how\PYZus{}many}\PY{p}{(}\PY{n}{Droid}\PY{p}{)}
\PY{n+nb}{print}\PY{p}{(}\PY{l+s+sa}{f}\PY{l+s+s1}{\PYZsq{}}\PY{l+s+s1}{У вас }\PY{l+s+si}{\PYZob{}}\PY{n}{droids\PYZus{}count}\PY{l+s+si}{\PYZcb{}}\PY{l+s+s1}{ дроидов.}\PY{l+s+s1}{\PYZsq{}}\PY{p}{)}
\end{Verbatim}
\end{tcolorbox}

    \begin{Verbatim}[commandchars=\\\{\}]
У вас 2 дроидов.
    \end{Verbatim}
\vfill
\end{frame}

\begin{frame}[fragile]{Переменные класса и объекта}
\scriptsize
    \begin{tcolorbox}[breakable, size=fbox, boxrule=1pt, pad at break*=1mm,colback=cellbackground, colframe=cellborder]
\prompt{In}{incolor}{4}{\boxspacing}
\begin{Verbatim}[commandchars=\\\{\}]
\PY{k}{class} \PY{n+nc}{Droid}\PY{p}{(}\PY{n+nb}{object}\PY{p}{)}\PY{p}{:}
    \PY{n}{population} \PY{o}{=} \PY{l+m+mi}{0}

    \PY{k}{def} \PY{n+nf+fm}{\PYZus{}\PYZus{}init\PYZus{}\PYZus{}}\PY{p}{(}\PY{n+nb+bp}{self}\PY{p}{,} \PY{n}{name}\PY{p}{)}\PY{p}{:}
        \PY{n+nb+bp}{self}\PY{o}{.}\PY{n}{name} \PY{o}{=} \PY{n}{name}
        \PY{n+nb}{print}\PY{p}{(}\PY{l+s+sa}{f}\PY{l+s+s1}{\PYZsq{}}\PY{l+s+s1}{  **Инициализация }\PY{l+s+si}{\PYZob{}}\PY{n+nb+bp}{self}\PY{o}{.}\PY{n}{name}\PY{l+s+si}{\PYZcb{}}\PY{l+s+s1}{**}\PY{l+s+s1}{\PYZsq{}}\PY{p}{)}
        \PY{n}{Droid}\PY{o}{.}\PY{n}{population} \PY{o}{+}\PY{o}{=} \PY{l+m+mi}{1}

    \PY{k}{def} \PY{n+nf}{say\PYZus{}hi}\PY{p}{(}\PY{n+nb+bp}{self}\PY{p}{)}\PY{p}{:}
        \PY{n+nb}{print}\PY{p}{(}\PY{l+s+sa}{f}\PY{l+s+s1}{\PYZsq{}}\PY{l+s+s1}{Приветствую! Мои хозяева называют меня }\PY{l+s+si}{\PYZob{}}\PY{n+nb+bp}{self}\PY{o}{.}\PY{n}{name}\PY{l+s+si}{\PYZcb{}}\PY{l+s+s1}{.}\PY{l+s+s1}{\PYZsq{}} \PY{p}{)}

    \PY{k}{def} \PY{n+nf+fm}{\PYZus{}\PYZus{}len\PYZus{}\PYZus{}}\PY{p}{(}\PY{n+nb+bp}{self}\PY{p}{)}\PY{p}{:}
        \PY{k}{return} \PY{n+nb+bp}{self}\PY{o}{.}\PY{n}{population}
\end{Verbatim}
\end{tcolorbox}
\vfill
\end{frame}


\begin{frame}[fragile]{Переменные класса и объекта}
\scriptsize
    \begin{tcolorbox}[breakable, size=fbox, boxrule=1pt, pad at break*=1mm,colback=cellbackground, colframe=cellborder]
\prompt{In}{incolor}{5}{\boxspacing}
\begin{Verbatim}[commandchars=\\\{\}]
\PY{n}{droid1} \PY{o}{=} \PY{n}{Droid}\PY{p}{(}\PY{l+s+s1}{\PYZsq{}}\PY{l+s+s1}{R2\PYZhy{}D2}\PY{l+s+s1}{\PYZsq{}}\PY{p}{)}
\PY{n}{droid1}\PY{o}{.}\PY{n}{say\PYZus{}hi}\PY{p}{(}\PY{p}{)}
\PY{n+nb}{print}\PY{p}{(}\PY{l+s+sa}{f}\PY{l+s+s1}{\PYZsq{}}\PY{l+s+s1}{У вас }\PY{l+s+si}{\PYZob{}}\PY{n+nb}{len}\PY{p}{(}\PY{n}{droid1}\PY{p}{)}\PY{l+s+si}{\PYZcb{}}\PY{l+s+s1}{ дроидов.}\PY{l+s+s1}{\PYZsq{}}\PY{p}{)}

\PY{n}{droid2} \PY{o}{=} \PY{n}{Droid}\PY{p}{(}\PY{l+s+s1}{\PYZsq{}}\PY{l+s+s1}{C\PYZhy{}3PO}\PY{l+s+s1}{\PYZsq{}}\PY{p}{)}
\PY{n}{droid2}\PY{o}{.}\PY{n}{say\PYZus{}hi}\PY{p}{(}\PY{p}{)}
\PY{n+nb}{print}\PY{p}{(}\PY{l+s+sa}{f}\PY{l+s+s1}{\PYZsq{}}\PY{l+s+s1}{У вас }\PY{l+s+si}{\PYZob{}}\PY{n+nb}{len}\PY{p}{(}\PY{n}{droid2}\PY{p}{)}\PY{l+s+si}{\PYZcb{}}\PY{l+s+s1}{ дроидов.}\PY{l+s+s1}{\PYZsq{}}\PY{p}{)}
\end{Verbatim}
\end{tcolorbox}

    \begin{Verbatim}[commandchars=\\\{\}]
  **Инициализация R2-D2**
Приветствую! Мои хозяева называют меня R2-D2.
У вас 1 дроидов.
  **Инициализация C-3PO**
Приветствую! Мои хозяева называют меня C-3PO.
У вас 2 дроидов.
    \end{Verbatim}
\vfill
\end{frame}

\subsection{Методы}
\begin{frame}[fragile]{Методы}
\scriptsize
Существуют три модификатора доступа:
\vfill
\begin{itemize}
\item private: \mintinline{python}|__x|
\item protected: \mintinline{python}|_x|
\item public: \mintinline{python}|x|
\end{itemize}
\vfill
\begin{minted}{python}
class Car:
    def __init__(self, engine, wheels=4):
        self.engine = engine
        self.wheels = wheels
        self.__speed = 0

    def go(self, acceleration_speed=20):
        if self.engine:
            self.__speed += acceleration_speed
            self.engine.go(self.__speed)
            print(self.status)
        else:
            print('Looks like you forgot to insert engine, '
                  'please do this before going anywhere.')
\end{minted}
\vfill
\end{frame}


\begin{frame}[fragile]{Методы}
\scriptsize
\begin{minted}[firstnumber=last]{python}
    def brake(self, braking_speed=20):
        self.__speed -= braking_speed
        self.engine.rotation -= braking_speed * 8
        print(self.status)

    def upgrade_engine(self, power):
        self.engine.power += power

    def get_status(self):
        return "We're driving so fast, " \
               f'our speed is {self.__speed}, rotation is {self.engine.rotation}'

    @property
    def status(self):
        return "We're driging so fast, " \
               f'our speed is {self.__speed}, rotation is {self.engine.rotation}'
\end{minted}
\vfill
\end{frame}

\begin{frame}[fragile]{Методы}
\scriptsize
\begin{minted}[firstnumber=last]{python}
class Engine:
    def __init__(self, power):
        self.power = power
        self.rotation = 0

    def start(self):
        self.rotation = 800

    def stop(self):
        self.rotation = 0

    def __eq__(self, other):
        if not isinstance(other, Engine):
            return False
        return self.power == other.power

    def go(self, speed):
        if not self.rotation:
            self.start()
        self.rotation = (3000 * speed) / self.power
\end{minted}
\vfill
\end{frame}

\begin{frame}[fragile]{Методы}
\scriptsize
\begin{minted}[firstnumber=last]{python}
if __name__ == '__main__':
    my_car = Car(engine=None)
    my_car.go(100)

    my_car.engine = Engine(100)
    my_car.go(100)
    my_car.go(200)

    my_car.upgrade_engine(50)
    my_car.go()

    my_car.upgrade_engine(100)
    my_car.go(0)
\end{minted}
\vfill
    \begin{Verbatim}[commandchars=\\\{\}]
Looks like you forgot to insert engine, please do this before going anywhere.
We're driging so fast, our speed is 100, rotation is 3000.0
We're driging so fast, our speed is 300, rotation is 9000.0
We're driging so fast, our speed is 320, rotation is 6400.0
We're driging so fast, our speed is 320, rotation is 3840.0
    \end{Verbatim}
\vfill
\end{frame}

\subsection{Наследование}
\begin{frame}[fragile]{Наследование}
\scriptsize
\textcolor{tpugreen}{\textbf{Наследование}}~-- концепция объектно-ориентированного программирования, согласно которой абстрактный тип данных может наследовать данные и функциональность некоторого существующего типа, способствуя повторному использованию его компонентов.
\vfill
\begin{minted}{python}
class Car:
    def __init__(self, engine_power, material='iron', wheels=4):
        self.engine_power = engine_power
        self.wheels = wheels
        self.material = material

        self._color = 'transparent'

    def paint(self, color):
        print('I do not know how to paint myself')

    def get_price(self):
        price = self.engine_power * 100 if self.engine_power else 0
        return price

    def __str__(self):
        return (f'Car ({self.get_price()}$): power: {self.engine_power},'
        f' material: {self.material}, color: {self._color}')
\end{minted}
\vfill
\end{frame}


\begin{frame}[fragile]{Наследование}
\scriptsize
\begin{minted}[firstnumber=last]{python}
class Toyota(Car):
    def __init__(self, engine_power, material='iron', **kwargs):
        super().__init__(engine_power, material='super iron', **kwargs)

    def paint(self, color):
        self._color = f'Toyota special "{color}"'

class Lexus(Toyota):
    def get_price(self):
        return super(Lexus, self).get_price() * 1.24

camry = Toyota(211)
camry.paint('black')
print('Camry', camry)

lx500 = Lexus(300)
lx500.paint('gray')
print('LX500', lx500)
\end{minted}
\vfill
    \begin{Verbatim}[commandchars=\\\{\}]
Camry Car (21100$): power: 211, material: super iron, color: Toyota special "black"
LX500 Car (37200.0$): power: 300, material: super iron, color: Toyota special "gray"
    \end{Verbatim}
\vfill
\end{frame}

\section{Пример создания класса}
\sectionframe

\begin{frame}[fragile]{Пример создания класса}
\scriptsize
Рассмотрим пример реализации класса \mintinline{python}|Flow|, являющегося представлением объекта материального потока.
\vfill
\begin{minted}{python}
import numpy as np

class Flow:
    def __init__(
        self,
        mass_flow_rate: float,
        mass_fractions: np.ndarray,
        temperature: float,
        pressure: float
    ) -> None:
        self.mass_flow_rate = mass_flow_rate
        self.mass_fractions = mass_fractions
        self.temperature = temperature
        self.pressure = pressure

    def convert_mass_to_volume_fractions(self, densities: np.ndarray) -> np.ndarray:
        x = self.mass_fractions / densities
        s = x.sum()
        return x / s
\end{minted}
\vfill
\end{frame}

\begin{frame}[fragile]{Пример создания класса}
\scriptsize
\begin{minted}[firstnumber=last]{python}
    def convert_mass_to_mole_fractions(self, mr: np.ndarray) -> np.ndarray:
        x = self.mass_fractions / mr
        s = x.sum()
        return x / s

    def get_flow_density(self, densities: np.ndarray) -> float:
        return (self.mass_fractions / densities).sum() ** -1

    def get_average_molar_mass(self, mr: np.ndarray) -> float:
        return (self.mass_fractions / mr).sum() ** -1


densities = np.array([.416, .546, .585, .5510, .6, .616, .6262, .6594])
mr = np.array([16, 30, 44, 58, 58, 72, 72, 86])
mass_fractions = np.array([.1, .1, .1, .4, .2, .05, .03, .02])

f1 = Flow(
    mass_flow_rate=100, mass_fractions=mass_fractions,
    temperature=150, pressure=101.325
)
\end{minted}
\vfill
\end{frame}

\begin{frame}[fragile]{Пример создания класса}
\scriptsize
\begin{minted}[firstnumber=last]{python}
f1.volume_fractions = f1.convert_mass_to_volume_fractions(densities)
f1.mole_fractions = f1.convert_mass_to_mole_fractions(mr)
f1.density = f1.get_flow_density(densities)
f1.mol_mass = f1.get_average_molar_mass(mr)

print(f1.volume_fractions)
print(f1.mole_fractions)
print(f1.density)
print(f1.mol_mass)
\end{minted}
\vfill
    \begin{Verbatim}[commandchars=\\\{\}]
[0.13257709 0.10101112 0.09427704 0.400378   0.18384023 0.04476629
 0.02642226 0.01672796]
[0.26545413 0.14157554 0.09652877 0.2929149  0.14645745 0.0294949
 0.01769694 0.00987736]
0.5515207014243224
42.47266071679689
    \end{Verbatim}
\vfill
\end{frame}


\contactsframe[\Large \textbf{Благодарю за внимание!}]{

	\bigskip
	\includegraphics[width=.05\textwidth]{pics/home} \quad Учебный корпус №2, ауд. 136 \\
	\includegraphics[width=.05\textwidth]{pics/mail} \quad chuva@tpu.ru \\
	\includegraphics[width=.03\textwidth]{pics/tel} \quad +7-962-782-66-15
}

\end{document}

