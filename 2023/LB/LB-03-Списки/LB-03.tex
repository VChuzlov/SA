
% !TeX document-id = {d8b4925c-2057-42a4-b894-2f1a3f1b6345}
%!TeX TXS-program:compile = txs:///xelatex/[--shell-escape]
\documentclass[aspectratio=169, mathserif]{beamer}% TPU recommends 16:9 ratio, 4:3 may require some work with inner theme .sty file

% Style options:
% light --- light theme (default)
% dark --- dark theme
% enlogo --- english TPU logo {default}
% rulogo --- russian TPU logo

\usetheme[light, rulogo]{tpu}% dark theme used as an example of optional argument

\usepackage[russian]{babel}%uncomment this to work in russian
\usepackage[utf8]{inputenc}

\usepackage{fontspec}

\setromanfont{Brygada1918}[
Path=./fonts/BrygadaFontFiles/,
Extension = .ttf,
UprightFont=*-Regular,
BoldFont=*-Bold,
ItalicFont=*-Italic,
BoldItalicFont=*-BoldItalic
]

\setsansfont{ALSSirius}[
Path=./fonts/ALSSiriusFiles/,
Extension = .otf,
UprightFont=*-Regular,
BoldFont=*-Bold,
%ItalicFont=*-Italic,
%BoldItalicFont=*-BoldItalic
]

\setmonofont{Consolas}[
Path=./fonts/ConsolasFontFiles/,
%Scale=0.85,
Extension = .ttf,
UprightFont=*-Regular,
BoldFont=*-Bold,
ItalicFont=*-Italic,
BoldItalicFont=*-BoldItalic
]

\usepackage[cache=false]{minted}
\usepackage{xcolor} % to access the named colour LightGray
\definecolor{LightGray}{gray}{0.9}
\definecolor{onedarkBckGr}{RGB}{40, 44, 52}

\usemintedstyle[python]{default}
\setminted[python]{
fontsize=\scriptsize,
escapeinside=||,
mathescape=true,
numbersep=5pt,
gobble=2,
linenos=false,
frame=single,
framesep=1mm,
python3=true,
bgcolor=backcolour,
}

\usemintedstyle[pycon]{default}
\setminted[pycon]{
	fontsize=\scriptsize,
	escapeinside=||,
	mathescape=true,
	numbersep=5pt,
	gobble=2,
	linenos=false,
	frame=single,
	framesep=1mm,
	python3=true,
	bgcolor=backcolour,
	linenos=true,
}

\newmint{python}{}

\usepackage{booktabs}% good looking tables
\usepackage{multicol}% text in multiple columns, useful for side-by-side text and pictures
\usepackage{hyperref}
\definecolor{maroon}{cmyk}{0, 0.87, 0.68, 0.32}
\definecolor{halfgray}{gray}{0.55}
\definecolor{ipython_frame}{RGB}{207, 207, 207}
\definecolor{ipython_bg}{RGB}{247, 247, 247}
\definecolor{ipython_red}{RGB}{186, 33, 33}
\definecolor{ipython_green}{RGB}{0, 128, 0}
\definecolor{ipython_cyan}{RGB}{64, 128, 128}
\definecolor{ipython_purple}{RGB}{170, 34, 255}
\definecolor{linkcolor}{HTML}{0000FF} % цвет гиперссылок
\definecolor{urlcolor}{HTML}{800080} % цвет ссылок
\definecolor{backcolour}{rgb}{0.95,0.95,0.92}

\usepackage{longtable}
\usepackage{wrapfig}
\usepackage{ragged2e}
\usepackage[nooneline]{caption}
\DeclareCaptionTextFormat{center}{\centering{#1}}
\captionsetup[table]{justification=raggedleft, 
labelformat=empty,
labelsep=endash,  
textformat=center, 
position=top, 
skip=5pt
}

\hyphenpenalty=10000% i don’t think hyphenation in presentations is a good idea, feel free to change however you like

\title{\LARGE{Системный анализ процессов химической технологии}}
\subtitle{Лабораторная работа №3 \\ Структуры данных: списки}
\author[]{Вячеслав Алексеевич Чузлов, \\
к.т.н., доцент ОХИ ИШПР}
\date{\today}

\begin{document}

\newcommand{\pythoninline}[1]{%
	\colorbox{white}{%
		\parbox[b][.6em]{\widthof{\mintinline[fontsize=\tiny]{ipython}{#1}}}{\mintinline[fontsize=\tiny]{ipython}{#1}}%
	}%
}

% notice usage of \titleframe and several other unconventional functions
% the reason being is custom backgrounds on these slides

\titleframe% title

%\tocframe{}% this custom frame accepts options for ToC



\section{Списки}
\sectionframe


\begin{frame}[fragile]{Списки}
\scriptsize
\begin{itemize}
\item Списки в Python~-- наиболее гибкая разновидность объектов упорядоченных коллекций. 

\item Списки могут содержать объекты любого типа: строки, числа или другие списки.  

\item Списки \textcolor{extraorange}{\textbf{можно изменять}} на месте присваиванием по индексам, с использованием срезов, вызвав специальные методы или выполнив оператор удаления. 
\end{itemize}

\begin{table}[h!]
\centering
\tiny
\begin{tabular}{|p{0.52\linewidth}|p{0.44\linewidth}|}
\hline
\textbf{Операция} & \textbf{Описание} \\
\hline
\mintinline{ipython}|a = []| & Пустой список \\
%\hline
\mintinline{ipython}|a = [123, 'abc', 1.354, []]| & Четыре элемента: индексы 0...3 \\
%\hline
\mintinline{ipython}|a = ['Joe', 30.0, ['dev', 'prof']]| & Вложенные списки \\
%\hline
\mintinline{ipython}|a = list('hello')| & Список элементов итерируемого объекта \\
%\hline
\mintinline{ipython}|a = list(range(-5, 6))| & Список последовательных целых чисел \\
%\hline
\mintinline{ipython}|a[i]| & Индекс \\
%\hline
\mintinline{ipython}|a[i][j]| & Индекс индекса \\
%\hline
\mintinline{ipython}|a[i:j]| & Срез \\
%\hline
\mintinline{ipython}|len(a)| & Длина \\
%\hline
\mintinline{ipython}|a1 + a2| & Конкатенация \\
%\hline
\mintinline{ipython}|a * 3| & Повторение \\
%\hline
\mintinline{ipython}|x in a| & Вхождение \\
%\hline
\mintinline{ipython}|a.append(5)| & Добавление элемента в конец списка \\
%\hline
\mintinline{ipython}|a.extend([10, 20, 30])| & Добавление списка в конец исходного списка \\
\hline
\end{tabular}

\end{table}
\vfill
\end{frame}


\subsection{Базовые операции со списками}

\begin{frame}[fragile]{Базовые операции со списками}
\scriptsize
\begin{itemize}
\item Списки являются \textcolor{extraorange}{\textbf{последовательностями}}, поэтому поддерживают многие операции, характерные для строк. 

\item Например, для списков определены операторы \pythoninline{+} и \pythoninline{*}. Данные операторы, также как и в случае со строками, означают конкатенацию и повторение, только возвращают в качестве результата новый список, а не строку.
\end{itemize}

\begin{minted}{pycon}
>>> len([1, 2, 3, 4, 5])                # Длина
5

>>> [1, 2, 3, 4, 5] + [6, 7, 8, 9, 10]  # Конкатенация
[1, 2, 3, 4, 5, 6, 7, 8, 9, 10]

>>> ['Hi!'] * 5                         # Повторение
['Hi!', 'Hi!', 'Hi!', 'Hi!', 'Hi!']
\end{minted}
\vfill
\end{frame}


\subsection{Итерация по спискам и генераторы списков}

\begin{frame}[fragile]{Итерация по спискам}
\scriptsize
В общем смысле для списков определены все операции над последовательностями, в том числе и инструменты итерации:

\begin{minted}{pycon}
>>> 'banana' in ['banana', 'orange', 'apple']   # Проверка вхождения
True

>>> for fruit in ['banana', 'orange', 'apple']: # Итерация
...     print(fruit, end=' ')
...
banana orange apple
\end{minted}

Оператор цикла \mintinline{python}|for| проходит (итерируется) по всем элементам в любой последовательности (итерируемом объекте) слева направо, выполняя операторы для каждого из них.
\vfill
\end{frame}


\begin{frame}[fragile]{Генераторы списков (list comprehension)}
\scriptsize
\textcolor{extraorange}{\textbf{Генераторы списков}}~-- это способ создания нового списка с применением выражения к каждому элементу последовательности (по факту в любом итерируемом объекте).

\begin{minted}{pycon}
>>> res = [c * 4 for c in 'HELLO']

>>> res
['HHHH', 'EEEE', 'LLLL', 'LLLL', 'OOOO']
\end{minted}

\begin{itemize}
\item Генераторы списков записываются более кратко и выполняются чуть быстрее.

\item В сложных случаях лучше использовать цикл \mintinline{python}|for| из-за его более высокой читаемости.
\end{itemize}

\begin{minted}[firstnumber=last]{pycon}
>>> res = []

>>> for c in 'HELLO':
...     res.append(c * 4)
...

>>> res
['HHHH', 'EEEE', 'LLLL', 'LLLL', 'OOOO']
\end{minted}
\vfill
\end{frame}


\subsection{Индексация, срезы и вложенность}

\begin{frame}[fragile]{Индексация и срезы}
\scriptsize
\begin{itemize}
\item Индексация  и срезы для списков работают аналогично тому, как это было описано для объектов строк. 

\item Результатом индексации списка может быть объект любого типа, находящийся по указанному индексу, тогда как  срезы всегда возвращают новый объект списка.


\begin{minted}{pycon}
>>> fruits = ['banana', 'orange', 'apple']

>>> fruits[2]            # Индексы начинаются с нуля
'apple'

>>> fruits[-2]           # Отрицательные индексы отсчитываются справа
'orange'

>>> fruits[1:3]          # Срезы получают сегменты
['orange', 'apple']

>>> fruits[-1]           # Результат среза всегда новый список
['apple']
\end{minted}
\end{itemize}
\vfill
\end{frame}


\begin{frame}[fragile]{Вложенность списков}
\scriptsize
\begin{itemize}
\item Внутри списков могут содержаться вложенные списки или объекты других типов.
\item Матрицы в Python можно представить в виде вложенных списков.
Пример матрицы $3 \times 3$:

\begin{minted}{pycon}
>>> matrix = [[10, 20, 30], [40, 50, 60], [70, 80, 90]]
\end{minted}
\item Если указать один индекс, то будет получена целая строка, а при указании двух индексов будет возвращен элемент строки:

\begin{minted}[firstnumber=last]{pycon}
>>> matrix[1]
[40, 50, 60]

>>> matrix[2][0]
70

>>> matrix = [[10, 20, 30],
...           [40, 50, 60],
...           [70, 80, 90]]

>>> matrix[1][1]
50
\end{minted}
\end{itemize}
\vfill
\end{frame}


\subsection{Изменение списков}

\begin{frame}[fragile]{Изменение списков}
\scriptsize
\begin{itemize}
\item Так как списки~-- \textcolor{extraorange}{\textbf{изменяемый}} тип объектов, для них определены операции, которые могут модифицировать объект списка \textcolor{extraorange}{\textbf{на месте}}.

\item Операции модифицирования списков изменяют объект списка напрямую, перезаписывая его старое значение, без необходимости создания новой копии, как в случае работы со строками.
\end{itemize}

\alert{\textbf{Присваивание по индексам и срезам}}
\\
Содержимое списка может быть изменено присваиванием значения либо отдельному элементу (по его индексу), либо целому сегменту (по срезу):

\begin{minted}{pycon}
>>> food = ['burger', 'pizza', 'buritto']

>>> food[1] = 'toast'  # Присваивание по индексу

>>> food
['burger', 'toast', 'buritto']

>>> food[:2] = ['need', 'more']  # Присваивание срезу

>>> food
['need', 'more', 'buritto']
\end{minted}
\vfill
\end{frame}


\subsection{Вызовы методов списков}

\begin{frame}[fragile]{Вызовы методов списков}
\scriptsize
\begin{itemize}
\item Подобно строкам, списки имеют набор специфичных методов, многие из которых ведут к изменению исходного списка на месте:
\begin{minted}{pycon}
>>> a = ['eat', 'more', 'SPAM']

>>> a.append('please')

>>> a
['eat', 'more', 'SPAM', 'please']

>>> a.sort()  # Сортировка элементов списка ('S' < 'e')
   
>>> a
['SPAM', 'eat', 'more', 'please']
\end{minted}
\item Наиболее распространенный метод~-- \mintinline{python}|append()| добавляет объект в конец списка. 
\item Эффект выполнения выражения \mintinline{python}|a.append(x)| аналогичен \mintinline{python}|a + [x]| с одним принципиальным отличием: первый вариант изменяет \texttt{a} на месте, а второй вариант создает новый объект списка. 
\item Метод \mintinline{python}|sort()| упорядочивает элементы в списке.
\end{itemize}
\vfill
\end{frame}

\begin{frame}[fragile]{Дополнительные сведения о методе \texttt{sort()}}
\scriptsize
\begin{itemize}
	\item В методе \mintinline{python}|sort()| аргумент \mintinline{python}|reverse| позволяет производить сортировку в порядке убывания вместо возрастания, а параметр \mintinline{python}|key| задает функцию с одним аргументом, которая возвращает значение для использования при сортировке.
\begin{minted}[fontsize=\fontsize{8pt}{8pt}]{pycon}
>>> a = ["abc", "ABD", "aBe"]   

>>> a.sort()                             # Сортировка со смешанным регистром

>>> a
['ABD', 'aBe', 'abc']

>>> a = ["abc", "ABD", "aBe"]

>>> a.sort(key=str.lower)                # Приведение к нижнему регистру

>>> a
['abc', 'ABD', 'aBe']

>>> a.sort(key=str.lower, reverse=True)  # Изменение порядка сортировки

>>> a
['aBe', 'ABD', 'abc']
\end{minted}
\end{itemize}
\vfill
\end{frame}


\section{Пример}
\sectionframe

\begin{frame}[fragile]{\space}
\scriptsize
\begin{alertblock}{\textbf{Пример}}
По имеющимся исходным данным определите состав потока в объемных и мольных долях, используя следующие формулы:
\end{alertblock}
\vfill
\begin{large}
\begin{equation*}
		\varphi _i = \dfrac{\dfrac{\omega _i}{\rho _i}}{\sum \limits _{i=1}^{n} \dfrac{\omega _i}{\rho _i}} 
\end{equation*}
\end{large}
\vfill
где $\varphi _i$~-- объемная доля $i$-го компонента; $\omega _i$~-- массовая доля $i$-го компонента; $\rho _i$~-- плотность $i$-го компонента; $n$~-- число компонентов в системе; $i$~-- индекс компонента в системе.
\vfill
\end{frame}

\begin{frame}[fragile]{\space}
\scriptsize
\begin{alertblock}{\textbf{Пример}}
Исходные данные:
\end{alertblock}
\begin{table}[h!]
	\begin{tabular}{|p{.2\linewidth}|p{.05\linewidth}|p{.05\linewidth}|p{.05\linewidth}|p{.05\linewidth}|p{.05\linewidth}|p{.05\linewidth}|p{.05\linewidth}|p{.05\linewidth}|}
		\hline
		$\mathrm{Параметр}$ & $C_1$ & $C_2$ & $C_3$ & $iC_4$ & $nC_4$ & $iC_5$ & $nC_5$ & $C_6$ \\
		\hline
		$\omega _i$ & $0.1$ & $0.1$ & $0.1$ & $0.4$ & $0.2$ & $0.05$ & $0.03$ & $0.02$ \\
		\hline
		$\rho_i, \mathrm{г/см}^3$ & $0.416$ & $0.546$ & $0.585$ & $0.5510$ & $0.6$ & $0.616$ & $0.6262$ & $0.6594$\\
		\hline
		$M_i, \mathrm{г/моль}$ & $16$ & $30$ & $44$ & $58$ & $58$ & $72$ & $72$ & $86$ \\
		\hline
	\end{tabular}
\end{table}
\vfill
\end{frame}

\begin{frame}[fragile]{\space}
\scriptsize
\begin{alertblock}{\textbf{Пример}}
\begin{minted}[linenos=true, fontsize=\fontsize{8pt}{9pt}]{python}
mass_fractions = [.1, .1, .1, .4, .2, .05, .03, .02]
densities = [0.416, 0.546, 0.585, 0.5510, 0.6, 0.616, 0.6262, 0.6594]
mol_mass_list = [16, 30, 44, 58, 58, 72, 72, 86]

volume_fractions = [mf / d for mf, d in zip(mass_fractions, densities)]
s = sum(volume_fractions)
volume_fractions = [vof / s for vof in volume_fractions]

volume_fractions = []  # Alternative
for mf, d in zip(mass_fractions, densities):
    volume_fractions.append(mf / d)

s = sum(volume_fractions)
for i in range(len(volume_fractions)):
    volume_fractions[i] /= s
    
for vof in volume_fractions:  # вывод результата
    print(f'{vof:.4f}', end=' ')
\end{minted}
\mintinline{ipython}|0.1326 0.1010 0.0943 0.4004 0.1838 0.0448 0.0264 0.0167|
\end{alertblock}
\vfill
\end{frame}


\section{Задания}
\sectionframe

\begin{frame}[fragile]
\scriptsize
\begin{alertblock}{\textbf{Задание 1}}
Используя исходные данные из примера, рассчитайте:
\end{alertblock}
\vfill
\begin{enumerate}
	\item Состав потока в мольных долях: 
	$$
		\chi _i = \dfrac{\dfrac{\omega _i}{M_i}}{\sum \limits_{i=1}^{n}\dfrac{\omega _i}{M_i}}
	$$
	где $\chi _i$~-- мольная доля $i$-го компонента; $\omega _i$~-- массовая доля $i$-го компонента; $M_i$~-- молярная масса $i$-го компонента; $n$~-- число компонентов в системе; $i$~-- индекс компонента в системе.
	\item Плотность потока:
	$$
		\rho = \dfrac{1}{\sum \limits_{i=1}^{n}\dfrac{\omega_i}{\rho_i}}
	$$
	где $\rho$~-- плотность потока; $\omega _i$~-- массовая доля $i$-го компонента; $\rho _i$~-- плотность $i$-го компонента; $n$~-- число компонентов в системе; $i$~-- индекс компонента в системе.
	\item Среднюю молекулярную массу потока:
	$$
		m = \dfrac{1}{\sum \limits_{i=1}^{n}\dfrac{\omega_i}{M_i}}
	$$
	где $m$~-- средняя молекулярная масса потока; $\omega _i$~-- массовая доля $i$-го компонента; $M_i$~-- молярная масса $i$-го компонента; $n$~-- число компонентов в системе; $i$~-- индекс компонента в системе.
\end{enumerate}
\vfill
\end{frame}

\begin{frame}[fragile]{\space}
\scriptsize
\begin{alertblock}{\textbf{Задание 2}}
Пусть на смешение поступают материальные потоки следющего состава (массовые доли):
\end{alertblock}
\vfill
\begin{table}[h!]
	\begin{tabular}{|p{.2\linewidth}|p{.05\linewidth}|p{.05\linewidth}|p{.05\linewidth}|p{.05\linewidth}|p{.05\linewidth}|p{.05\linewidth}|p{.05\linewidth}|p{.05\linewidth}|}
		\hline
		$\mathrm{Поток}$ & $C_1$ & $C_2$ & $C_3$ & $iC_4$ & $nC_4$ & $iC_5$ & $nC_5$ & $C_6$ \\
		\hline
		$1$ & $0.1$ & $0.1$ & $0.1$ & $0.4$ & $0.2$ & $0.05$ & $0.03$ & $0.02$ \\
		\hline
		$2$ & $0.1$ & $0.2$ & $0.1$ & $0.3$ & $0.1$ & $0.15$ & $0.03$ & $0.02$ \\
		\hline
		$3$ & $0.1$ & $0.1$ & $0.15$ & $0.35$ & $0.1$ & $0.05$ & $0.08$ & $0.07$ \\
		\hline
	\end{tabular}
\end{table}
\vfill
Расходы потоков $100$, $150$ и $120$ моль/ч, соответственно.
Необходимо рассчитать состав итогового потока в массовых и мольных долях.

Состав смесевого потока (в массовых долях) можно найти следующим образом:
$$
	\dfrac{\sum\limits_{j=1}^{n}G_j \cdot \omega_{i, j}}{\sum\limits_{j=1}^{n}G_j}
$$
\vfill
где $\omega_i$~-- массовая доля $i$-го компонента в смесевом потоке; $\omega_{i, j}$~-- массовая доля $i$-го компонента в $j$-ом потоке; $G_j$~-- массовый расход $j$-го потока; $j$~-- индекс потока; $i$~-- индекс компонента в системе; $n$~-- число потоков, подаваемых на смешение.
\vfill
\end{frame}

\contactsframe[\Large \textbf{Благодарю за внимание!}]{

\includegraphics[width=.05\textwidth]{pics/home} \quad Учебный корпус №2, ауд. 136 \\
\includegraphics[width=.05\textwidth]{pics/mail} \quad chuva@tpu.ru \\
\includegraphics[width=.03\textwidth]{pics/tel} \quad +7-962-782-66-15
}

\end{document}

