%\part{Программирование на языке Python}


%\chapter{Архитектура программного решения}\label{ch:архитектура}
%%\addcontentsline{toc}{chapter}{Расчет физико-химических свойств}


%\begin{listing}[h]
%\caption{Описание кинетики с помощью матрицы (файл \texttt{kinetic.py})}
%\inputminted[label=Описание кинетики с помощью матрицы (файл kinetic.py)]{python}{../scripts/kinetic.py}
%\end{listing}
%
%\newpage
%\begin{listing}[h!]
%\caption{Подбор кинетических параметров с помощью методов оптимизации функций нескольких переменных (файл \texttt{main.py})}
%\inputminted[label=Подбор кинетических параметров с помощью методов оптимизации функций нескольких переменных (файл main.py), ]{python}{../scripts/main.py}
%\end{listing}
%
%\newpage
%\begin{listing}[thp]
%\caption{Исходные данные (файл \texttt{data.txt})}
%\inputminted{text}{../scripts/data.txt}
%\end{listing}
%
%\newpage
\lstinputlisting[language=iPython, caption={Описание кинетики с помощью матрицы (файл \texttt{kinetic.py})}, numbers=left]{../scripts/kinetic.py}
%
\newpage
\lstinputlisting[language=iPython, caption={Подбор кинетических параметров с помощью методов оптимизации функций нескольких переменных (файл \texttt{main.py})}, numbers=left]{../scripts/main.py}
%
\newpage
\lstinputlisting[language=TeX, caption={Исходные данные (файл \texttt{data.txt})}, ]{../scripts/data.txt}
%
%\newpage
%\lstinputlisting[language=iPython, caption={Описание класса \texttt{Splitter} (\texttt{splitter.py})}, numbers=left]{splitter.py}
