
% !TeX document-id = {d8b4925c-2057-42a4-b894-2f1a3f1b6345}
%!TeX TXS-program:compile = txs:///xelatex/[--shell-escape]
\documentclass[aspectratio=169, mathserif]{beamer}% TPU recommends 16:9 ratio, 4:3 may require some work with inner theme .sty file

% Style options:
% light --- light theme (default)
% dark --- dark theme
% enlogo --- english TPU logo {default}
% rulogo --- russian TPU logo

\usetheme[light, rulogo]{tpu}% dark theme used as an example of optional argument

\usepackage[russian]{babel}%uncomment this to work in russian
\usepackage[utf8]{inputenc}

\usepackage{fontspec}

\setromanfont{Brygada1918}[
Path=./fonts/BrygadaFontFiles/,
Extension = .ttf,
UprightFont=*-Regular,
BoldFont=*-Bold,
ItalicFont=*-Italic,
BoldItalicFont=*-BoldItalic
]

\setsansfont{ALSSirius}[
Path=./fonts/ALSSiriusFiles/,
Extension = .otf,
UprightFont=*-Regular,
BoldFont=*-Bold,
%ItalicFont=*-Italic,
%BoldItalicFont=*-BoldItalic
]

\setmonofont{Consolas}[
Path=./fonts/ConsolasFontFiles/,
%Scale=0.85,
Extension = .ttf,
UprightFont=*-Regular,
BoldFont=*-Bold,
ItalicFont=*-Italic,
BoldItalicFont=*-BoldItalic
]

\usepackage[cache=false]{minted}
\usepackage{xcolor} % to access the named colour LightGray
\definecolor{LightGray}{gray}{0.9}
\definecolor{onedarkBckGr}{RGB}{40, 44, 52}

\usemintedstyle[python]{default}
\setminted[python]{
fontsize=\scriptsize,
escapeinside=||,
mathescape=true,
numbersep=5pt,
gobble=2,
linenos=true,
frame=leftline,
framesep=1mm,
python3=true,
%bgcolor=backcolour,
}

\usemintedstyle[pycon]{default}
\setminted[pycon]{
	fontsize=\scriptsize,
	escapeinside=||,
	mathescape=true,
	numbersep=5pt,
	gobble=2,
	frame=single,
	framesep=1mm,
	python3=true,
%	bgcolor=backcolour,
	linenos=true,
}

\newmint{python}{}

\usepackage{booktabs}% good looking tables
\usepackage{multicol}% text in multiple columns, useful for side-by-side text and pictures
\usepackage{hyperref}
\definecolor{maroon}{cmyk}{0, 0.87, 0.68, 0.32}
\definecolor{halfgray}{gray}{0.55}
\definecolor{ipython_frame}{RGB}{207, 207, 207}
\definecolor{ipython_bg}{RGB}{247, 247, 247}
\definecolor{ipython_red}{RGB}{186, 33, 33}
\definecolor{ipython_green}{RGB}{0, 128, 0}
\definecolor{ipython_cyan}{RGB}{64, 128, 128}
\definecolor{ipython_purple}{RGB}{170, 34, 255}
\definecolor{linkcolor}{HTML}{0000FF} % цвет гиперссылок
\definecolor{urlcolor}{HTML}{800080} % цвет ссылок
\definecolor{backcolour}{rgb}{0.95,0.95,0.92}

\usepackage{longtable}
\usepackage{wrapfig}
\usepackage{ragged2e}
\usepackage[nooneline]{caption}
\DeclareCaptionTextFormat{center}{\centering{#1}}
\captionsetup[table]{justification=raggedleft,
labelformat=empty,
labelsep=endash,
textformat=center,
position=top,
skip=5pt
}

\hyphenpenalty=10000% i don’t think hyphenation in presentations is a good idea, feel free to change however you like

\usepackage{chemfig}
%\includeonlyframes{c}

\title{\LARGE{Системный анализ процессов химической технологии}}
\subtitle{\textcolor{tpugreen}{\textbf{Лабораторная работа №8}} \\ \textbf{Идентификация кинетических параметров \\ при моделировании химических реакций}}
\author[]{Вячеслав Алексеевич Чузлов, \\
к.т.н., доцент ОХИ ИШПР}
\date{\today}

\begin{document}

\newcommand{\pythoninline}[1]{%
	\colorbox{white}{%
		\parbox[b][.6em]{\widthof{\mintinline[fontsize=\tiny]{ipython}{#1}}}{\mintinline[fontsize=\tiny]{ipython}{#1}}%
	}%
}

% notice usage of \titleframe and several other unconventional functions
% the reason being is custom backgrounds on these slides

\titleframe% title

%\tocframe{}% this custom frame accepts options for ToC


\subsection{Задача}
\begin{frame}[fragile, label=c]{Задача}
\scriptsize
Необходимо определить кинетические параметры изменения концентрации каждого компонента в течение $1$ часа с шагом $0.1$. Концентрация $\left[C_9H_{20}\right]$ в начальный момент времени $1\ \mathrm{моль / л}$, концентрации остальных компонентов равны нулю. Построить зависимость $C\left(t\right)$ для каждого компонента.
\vfill
\textbf{Схема химических превращений:}
\vfill
\scalebox{.7}{
\setchemfig{scheme debug=false}
\schemestart \chemfig{H_3C-[:30]CH_2-[:-30]CH_2-[:30]CH_2-[:-30]CH_2-[:30]CH_2-[:-30]CH_2-[:30]CH_2-[:-30]CH_3} \arrow{->[$k_1$][]}\chemfig{H_2C=[:30]CH-[:-30]CH_2-[:30]CH_2-[:-30]CH_2-[:30]CH_2-[:-30]CH_2-[:30]CH_2-[:-30]CH_3} \+  \chemfig{H_2} \schemestop
}
\vfill
\scalebox{.7}{
\setchemfig{scheme debug=false}
\schemestart \chemfig{H_2C=[:30]CH-[:-30]CH_2-[:30]CH_2-[:-30]CH_2-[:30]CH_2-[:-30]CH_2-[:30]CH_2-[:-30]CH_3} \arrow{->[$k_2$][]}\chemfig{H_2C=[:30]CH-[:-30]CH=[:30]CH-[:-30]CH_2-[:30]CH_2-[:-30]CH_2-[:30]CH_2-[:-30]CH_3} \+  \chemfig{H_2} \schemestop
}
\vfill
\end{frame}

\begin{frame}[fragile, label=c]{Задача}
\scriptsize
\begin{minipage}{.3\textwidth}
$$
\left\{
\begin{aligned}
\dfrac{d\left[C_9H_{20}\right]}{dt} &= -k_1 \cdot \left[C_9H_{20}\right] \\
\dfrac{d\left[C_9H_{18}\right]}{dt} &= k_1 \cdot \left[C_9H_{20}\right] - k_2 \cdot \left[C_9H_{18}\right] \\
\dfrac{d\left[C_9H_{16}\right]}{dt} &= k_2 \cdot \left[C_9H_{18}\right] \\
\dfrac{d\left[H_2\right]}{dt} &= k_1 \cdot \left[C_9H_{20}\right] + k_2 \cdot \left[C_9H_{18}\right] \\
\end{aligned}
\right.
$$
\vfill
\begin{table}[h!]
\begin{tabular}{|l|r|r|r|r|}
\hline
Реакция & $C_9H_{20}$ & $C_9H_{18}$ & $C_9H_{16}$ & $H_2$ \\
\hline
$r_1$ & $-1$ & $1$ & $0$ & $1$ \\
\hline
$r_2$ & $0$ & $-1$ & $1$ & $1$ \\
\hline
\end{tabular}
\end{table}
\vfill
\end{minipage}
\hspace{.19\textwidth}
\begin{minipage}{.5\textwidth}
\centering
Концентрация компонентов, $\mathrm{моль / л}$
\begin{table}[h!]
\begin{tabular}{|l|r|r|r|r|}
\hline
Время, ч & $C_9H_{20}$ & $C_9H_{18}$ & $C_9H_{16}$ & $H_2$ \\
\hline
$0.1$ & $0.8353$ & $0.1563$ & $0.0084$ & $0.1732$ \\
\hline
$0.2$ & $0.6977$ & $0.2715$ & $0.0308$ & $0.3331$ \\
\hline
$0.3$ & $0.5827$ & $0.3540$ & $0.0633$ & $0.4805$ \\
\hline
$0.4$ & $0.4868$ & $0.4104$ & $0.1028$ & $0.6161$ \\
\hline
$0.5$ & $0.4066$ & $0.4463$ & $0.1471$ & $0.7405$ \\
\hline
$0.6$ & $0.3396$ & $0.4662$ & $0.1942$ & $0.8546$ \\
\hline
$0.7$ & $0.2837$ & $0.4736$ & $0.2427$ & $0.9591$ \\
\hline
$0.8$ & $0.2369$ & $0.4716$ & $0.2915$ & $1.0545$ \\
\hline
$0.9$ & $0.1979$ & $0.4625$ & $0.3396$ & $1.1417$ \\
\hline
$1.0$ & $0.1653$ & $0.4481$ & $0.3866$ & $1.2213$ \\
\hline
\end{tabular}
\end{table}
\end{minipage}
\vfill
\end{frame}


\contactsframe[\Large \textbf{Благодарю за внимание!}]{

\includegraphics[width=.05\textwidth]{pics/home} \quad Учебный корпус №2, ауд. 136 \\
\includegraphics[width=.05\textwidth]{pics/mail} \quad chuva@tpu.ru \\
\includegraphics[width=.03\textwidth]{pics/tel} \quad +7-962-782-66-15
}

\end{document}

