
% !TeX document-id = {d8b4925c-2057-42a4-b894-2f1a3f1b6345}
%!TeX TXS-program:compile = txs:///xelatex/[--shell-escape]
\documentclass[aspectratio=169, mathserif]{beamer}% TPU recommends 16:9 ratio, 4:3 may require some work with inner theme .sty file

% Style options:
% light --- light theme (default)
% dark --- dark theme
% enlogo --- english TPU logo {default}
% rulogo --- russian TPU logo

\usetheme[light, rulogo]{tpu}% dark theme used as an example of optional argument

\usepackage[russian]{babel}%uncomment this to work in russian
\usepackage[utf8]{inputenc}

\usepackage{fontspec}

\setromanfont{Brygada1918}[
Path=./fonts/BrygadaFontFiles/,
Extension = .ttf,
UprightFont=*-Regular,
BoldFont=*-Bold,
ItalicFont=*-Italic,
BoldItalicFont=*-BoldItalic
]

\setsansfont{ALSSirius}[
Path=./fonts/ALSSiriusFiles/,
Extension = .otf,
UprightFont=*-Regular,
BoldFont=*-Bold,
%ItalicFont=*-Italic,
%BoldItalicFont=*-BoldItalic
]

\setmonofont{Consolas}[
Path=./fonts/ConsolasFontFiles/,
%Scale=0.85,
Extension = .ttf,
UprightFont=*-Regular,
BoldFont=*-Bold,
ItalicFont=*-Italic,
BoldItalicFont=*-BoldItalic
]

\usepackage[cache=false]{minted}
\usepackage{xcolor} % to access the named colour LightGray
\definecolor{LightGray}{gray}{0.9}
\definecolor{onedarkBckGr}{RGB}{40, 44, 52}

\usemintedstyle[python]{default}
\setminted[python]{
fontsize=\scriptsize,
escapeinside=||,
mathescape=true,
numbersep=5pt,
gobble=2,
linenos=false,
frame=single,
framesep=1mm,
python3=true,
bgcolor=backcolour,
}

\usemintedstyle[pycon]{default}
\setminted[pycon]{
	fontsize=\scriptsize,
	escapeinside=||,
	mathescape=true,
	numbersep=5pt,
	gobble=2,
	linenos=false,
	frame=single,
	framesep=1mm,
	python3=true,
	bgcolor=backcolour,
	linenos=true,
}

\newmint{python}{}

\usepackage{booktabs}% good looking tables
\usepackage{multicol}% text in multiple columns, useful for side-by-side text and pictures
\usepackage{hyperref}
\definecolor{maroon}{cmyk}{0, 0.87, 0.68, 0.32}
\definecolor{halfgray}{gray}{0.55}
\definecolor{ipython_frame}{RGB}{207, 207, 207}
\definecolor{ipython_bg}{RGB}{247, 247, 247}
\definecolor{ipython_red}{RGB}{186, 33, 33}
\definecolor{ipython_green}{RGB}{0, 128, 0}
\definecolor{ipython_cyan}{RGB}{64, 128, 128}
\definecolor{ipython_purple}{RGB}{170, 34, 255}
\definecolor{linkcolor}{HTML}{0000FF} % цвет гиперссылок
\definecolor{urlcolor}{HTML}{800080} % цвет ссылок
\definecolor{backcolour}{rgb}{0.95,0.95,0.92}

\usepackage{longtable}
\usepackage{wrapfig}
\usepackage{ragged2e}
\usepackage[nooneline]{caption}
\DeclareCaptionTextFormat{center}{\centering{#1}}
\captionsetup[table]{justification=raggedleft, 
labelformat=empty,
labelsep=endash,  
textformat=center, 
position=top, 
skip=5pt
}

\hyphenpenalty=10000% i don’t think hyphenation in presentations is a good idea, feel free to change however you like

\title{\LARGE{Системный анализ процессов химической технологии}}
\subtitle{Лабораторная работа №4 \\ Функции}
\author[]{Вячеслав Алексеевич Чузлов, \\
к.т.н., доцент ОХИ ИШПР}
\date{\today}

\begin{document}

\newcommand{\pythoninline}[1]{%
	\colorbox{white}{%
		\parbox[b][.6em]{\widthof{\mintinline[fontsize=\tiny]{ipython}{#1}}}{\mintinline[fontsize=\tiny]{ipython}{#1}}%
	}%
}

% notice usage of \titleframe and several other unconventional functions
% the reason being is custom backgrounds on these slides

\titleframe% title

%\tocframe{}% this custom frame accepts options for ToC



\section{Описание функций}
\sectionframe


\begin{frame}[fragile]{Написание кода функций}
\scriptsize
\begin{itemize}
\item \textcolor{extraorange}{\textbf{Функции}}~-- это многократно используемые фрагменты программы. Они позволяют дать имя определенному блоку команд с тем, чтобы в последствии запускать блок по указанному имени в любом месте программы и сколь угодно много раз. Это называется \textbf{вызовом функции}.
\item Функции определяются при помощи зарезервированного слова \mintinline{python}|def|. После этого слова указывается имя функции, за которым следует пара скобок, в которых можно указать имена некоторых переменных, и заключительное двоеточие в конце строки. Далее следует блок команд (инструкций), составляющих тело функции.

\item \textcolor{extraorange}{\textbf{Сигнатура функции}}~-- часть общего объявления функции, позволяющая средствами трансляции идентифицировать функцию среди других. Составляющие сигнатуры:
\begin{enumerate}
\scriptsize
\item имя функции;
\item аргументы функции;
\item возвращаемые значения.
\end{enumerate} 

\begin{minted}{pycon}
>>> def say_hi():
...     print("Hi!")       

>>> say_hi()
Hi!
\end{minted}

\end{itemize}
\vfill
\end{frame}

\begin{frame}[fragile]{Оператор \texttt{return}}
\scriptsize
\begin{itemize}
\item Тело функции почти всегда содержит оператор \mintinline{python}|return|:
\begin{minted}{ipython}
def имя_функции(аргумент1, аргумент2, ..., аргумент3):
    операторы
    ...
    return ...
\end{minted}
\item Оператор \mintinline{python}|return| в Python может появляться где угодно в теле функции; по достижении он заканчивает выполнение функции и возвращает результат обратно вызывающему коду. 
\item Оператор \mintinline{python}|return| состоит из необязательного выражения с объектным значением, которое дает результат функции. 
\item Если значение опущено, тогда \mintinline{python}|return| возвращает \mintinline{python}|None|. 
\item Оператор \mintinline{python}|return| сам по себе также необязателен; если он отсутствует, то выход из функции происходит, когда  интерпретатор достигает конца тела функции. Формально функция без оператора \mintinline{python}|return| автоматически возвращает объект \mintinline{python}|None|. 
\item Хорошим тоном является явное использование пустого оператора \mintinline{python}|return| для дополнительного пояснения того, что функция ничего не возвращает в качестве результата.
\end{itemize}
\vfill
\end{frame}

\subsection{Оператор \texttt{return}}

\begin{frame}[fragile]{Оператор \texttt{return}}
\scriptsize
\begin{itemize}
\item Оператор \mintinline{python}|return| используется для возврата из функции, т.е. для прекращения её работы и выхода из неё. При этом можно также вернуть некоторое значение из функции.
\item Оператор \mintinline{python}|return| в Python может появляться где угодно в теле функции; по достижении он \textbf{заканчивает} \textbf{выполнение функции} и возвращает результат обратно вызывающему коду.
\end{itemize}

\begin{minipage}{.47\textwidth}
\begin{minted}{pycon}
>>> def maximum(x, y): 
...     if x > y:
...         return x
...
...     elif x == y:
...         return 'Equals.' 
...     
...     else:
...         return y

>>> print(maximum(2, 3))
3
\end{minted}
\end{minipage}
\begin{minipage}{.05\textwidth}
\hspace{.1ex}
\end{minipage}
\begin{minipage}{.47\textwidth}
\begin{minted}{pycon}
>>> def maximum(x, y): 
...     if x > y:
...         return x
...
...     if x == y:
...         return 'Equals.' 
...
...     return y

>>> print(maximum(2, 3))
3
\end{minted}
\vspace{.08ex}
\end{minipage}
\vfill
\end{frame}


\subsection{Локальные переменные}

\begin{frame}[fragile]{Локальные переменные}
\scriptsize
\begin{itemize}
\item При объявлении переменных внутри определения функции, они никоим образом не связаны с другими переменными с таким же именем за пределами функции~-- т.е. имена переменных являются локальными в функции. 
\item Это называется \textcolor{extraorange}{\textbf{областью видимости переменной}}. Область видимости всех переменных ограничена блоком, в котором они объявлены, начиная с точки объявления имени.

\begin{minted}{pycon}
>>> x = 50

>>> def func(x):
...     print('x =', x)
...     x = 2
...     print('Replace x to', x)
...

>>> func(x)
x = 50
Replace x to 2

>>> print('x =', x)
x = 50
\end{minted}
\end{itemize}
\vfill
\end{frame}


\section{Синтаксис передачи аргументов}
\sectionframe

\begin{frame}[fragile]{Позиционные аргументы}
\scriptsize
\begin{itemize}
\item Если не использовать какой-то специальный синтаксис сопоставления, то Python будет сопоставлять имена по позиции слева направо подобно большинству других языков. Например, если Вы определили функцию, которая требует трех аргументов, тогда должны вызывать ее с тремя аргументами:

\begin{minted}{pycon}
>>> def f(x, y, z):
...     return x, y, z
...

>>> f(0, 1, 2)
(0, 1, 2)
\end{minted}
\item Здесь аргументы передаются по позиции~-- \texttt{x} соответствует \texttt{0}, \texttt{y}~-- \texttt{1} и \texttt{z}~-- \texttt{2}.
\end{itemize}
\vfill
\end{frame}

\begin{frame}[fragile]{Именованные параметры}
\scriptsize
\begin{itemize}
\item \textcolor{extraorange}{\textbf{Именованные}} аргументы делают возможным сопоставление по имени, а не по позиции.

\begin{minted}[firstnumber=last]{pycon}
>>> f(z=2, x=0, y=1)
(0, 1, 2)
\end{minted}

\item Здесь \texttt{z=2} означает передачу значения \texttt{2} аргументу по имени \texttt{z}. Когда применяются ключевые слова, порядок следования аргументов несущественен, т.к. они сопоставляются по имени. 
\item Разрешено комбинировать позиционные и ключевые аргументы. Сначала сопоставляются позиционные аргументы слева направо в заголовке, а затем ключевые аргументы по имени:

\begin{minted}[firstnumber=last]{pycon}
>>> f(0, z=2, y=1)
(0, 1, 2)
\end{minted}

\item Ключевые аргументы делают вызовы функций самодокументированными. Вызов функции:

\begin{minted}{python}
func(name='James', age=20, job='student')
\end{minted}

\noindent выглядит более значащим по сравнению с вызовом, содержащим три разделенных запятыми значения, особенно в крупных программах.
\end{itemize}
\vfill
\end{frame}


\begin{frame}[fragile]{Значения по умолчанию}
\scriptsize
\begin{itemize}
\item Стандартные значения позволяют делать некоторые аргументы функции необязательными. 
\item Если значение для аргумента не было передано, то используется стандартное значение.

\begin{minted}[fontsize=\fontsize{7.pt}{7pt}]{pycon}
>>> def f(x, y=1, z=2):  # Аргумент x обязательный
...     return x, y, z   # y и z необязательные
\end{minted}

\item При вызове такой функции обязательно нужно передать значение для \texttt{x} по позиции, либо по имени;  передача значений для \texttt{y} и \texttt{z} необязательна:

\begin{minted}[firstnumber=last, fontsize=\fontsize{7.pt}{7pt}]{pycon}
>>> f(0)
(0, 1, 2)
>>> f(x=0)
(0, 1, 2)
\end{minted}

\item В случае передачи двух значений стандартное значение получит только аргумент \texttt{z}, а при передаче трех значений стандартные значения не применяются:
\begin{minted}[firstnumber=last, fontsize=\fontsize{7.pt}{7pt}]{pycon}
>>> f(10, 20)
(10, 20, 2)
>>> f(10, 20, 30)
(10, 20, 30)
\end{minted}
\end{itemize}
\vfill
\end{frame}

\section{Пример}
\sectionframe


\begin{frame}[fragile]{\textcolor{tpugreen}{\textbf{Пример}}}
\scriptsize
По имеющимся исходным данным определите состав потока в объемных долях, используя следующию формулу:
\begin{normalsize}
\begin{equation*}
	\varphi _i = \dfrac{\dfrac{\omega _i}{\rho _i}}{\sum \limits _{i=1}^{n} \dfrac{\omega _i}{\rho _i}} 
\end{equation*}
\end{normalsize}
где $\varphi _i$~-- объемная доля $i$-го компонента; $\omega _i$~-- массовая доля $i$-го компонента; $\rho _i$~-- плотность $i$-го компонента; $n$~-- число компонентов в системе; $i$~-- индекс компонента в системе.
Исходные данные:
\begin{table}[h!]
	\begin{tabular}{|p{.2\linewidth}|p{.05\linewidth}|p{.05\linewidth}|p{.05\linewidth}|p{.05\linewidth}|p{.05\linewidth}|p{.05\linewidth}|p{.05\linewidth}|p{.05\linewidth}|}
		\hline
		$\mathrm{Параметр}$ & $C_1$ & $C_2$ & $C_3$ & $iC_4$ & $nC_4$ & $iC_5$ & $nC_5$ & $C_6$ \\
		\hline
		$\omega _i$ & $0.1$ & $0.1$ & $0.1$ & $0.4$ & $0.2$ & $0.05$ & $0.03$ & $0.02$ \\
		\hline
		$\rho_i, \mathrm{г/см}^3$ & $0.416$ & $0.546$ & $0.585$ & $0.5510$ & $0.6$ & $0.616$ & $0.6262$ & $0.6594$\\
		\hline
		$M_i, \mathrm{г/моль}$ & $16$ & $30$ & $44$ & $58$ & $58$ & $72$ & $72$ & $86$ \\
		\hline
	\end{tabular}
\end{table}
Вычисления необходимо реализовать в виде функции.
\vfill
\end{frame}

\begin{frame}[fragile]{\textcolor{tpugreen}{\textbf{Пример}}}
\scriptsize
\begin{minted}[linenos=true, fontsize=\fontsize{8pt}{8pt}]{python}
def mass_to_volume_fractions(
    mass_fractions: list[float],
    densities: list[float]
) -> list[float]:
    
    volume_fractions = [
        mf / d for mf, d in zip(mass_fractions, densities)
    ]
    s = sum(volume_fractions)
    volume_fractions = [v / s for v in volume_fractions]
    
    return volume_fractions


mass_fractions = [.1, .1, .1, .4, .2, .05, .03, .02]
densities = [0.416, 0.546, 0.585, 0.5510, 0.6, 0.616, 0.6262, 0.6594]
mol_mass_list = [16, 30, 44, 58, 58, 72, 72, 86]

volume_fractions = mass_to_volume_fractions(mass_fractions, densities)

for vf in volume_fractions: # 0.1326 0.1010 0.0943 0.4004 0.1838 0.0448 0.0264 0.0167
    print(f'{vf:.4f}', end=' ')  
\end{minted}
\vfill
\end{frame}


\section{Задания}
\sectionframe

\begin{frame}[fragile]
\scriptsize
\begin{alertblock}{\textbf{Задание 1}}
Используя исходные данные из примера, рассчитайте, реализовав соответствующие функции:
\end{alertblock}
\vfill
\begin{enumerate}
	\item Состав потока в мольных долях: 
	$$
		\chi _i = \dfrac{\dfrac{\omega _i}{M_i}}{\sum \limits_{i=1}^{n}\dfrac{\omega _i}{M_i}}
	$$
	где $\chi _i$~-- мольная доля $i$-го компонента; $\omega _i$~-- массовая доля $i$-го компонента; $M_i$~-- молярная масса $i$-го компонента; $n$~-- число компонентов в системе; $i$~-- индекс компонента в системе.
	\item Плотность потока:
	$$
		\rho = \dfrac{1}{\sum \limits_{i=1}^{n}\dfrac{\omega_i}{\rho_i}}
	$$
	где $\rho$~-- плотность потока; $\omega _i$~-- массовая доля $i$-го компонента; $\rho _i$~-- плотность $i$-го компонента; $n$~-- число компонентов в системе; $i$~-- индекс компонента в системе.
	\item Среднюю молекулярную массу потока:
	$$
		m = \dfrac{1}{\sum \limits_{i=1}^{n}\dfrac{\omega_i}{M_i}}
	$$
	где $m$~-- средняя молекулярная масса потока; $\omega _i$~-- массовая доля $i$-го компонента; $M_i$~-- молярная масса $i$-го компонента; $n$~-- число компонентов в системе; $i$~-- индекс компонента в системе.
\end{enumerate}
\vfill
\end{frame}

\begin{frame}[fragile]{\textcolor{tpugreen}{\textbf{Задание~2}}}
\scriptsize
Пусть на смешение поступают материальные потоки следующего состава (массовые доли):
\vfill
\begin{table}[h!]
	\begin{tabular}{|p{.2\linewidth}|p{.05\linewidth}|p{.05\linewidth}|p{.05\linewidth}|p{.05\linewidth}|p{.05\linewidth}|p{.05\linewidth}|p{.05\linewidth}|p{.05\linewidth}|}
		\hline
		$\mathrm{Поток}$ & $C_1$ & $C_2$ & $C_3$ & $iC_4$ & $nC_4$ & $iC_5$ & $nC_5$ & $C_6$ \\
		\hline
		$1$ & $0.1$ & $0.1$ & $0.1$ & $0.4$ & $0.2$ & $0.05$ & $0.03$ & $0.02$ \\
		\hline
		$2$ & $0.1$ & $0.2$ & $0.1$ & $0.3$ & $0.1$ & $0.15$ & $0.03$ & $0.02$ \\
		\hline
		$3$ & $0.1$ & $0.1$ & $0.15$ & $0.35$ & $0.1$ & $0.05$ & $0.08$ & $0.07$ \\
		\hline
	\end{tabular}
\end{table}
\vfill
Расходы потоков $200$, $250$ и $120$ кг/ч, соответственно.
Необходимо рассчитать состав итогового потока в массовых долях, реализовав соответствующую функцию.

Состав смесевого потока можно найти следующим образом:
$$
	\dfrac{\sum\limits_{j=1}^{n}G_j \cdot \omega_{i, j}}{\sum\limits_{j=1}^{n}G_j}
$$
\vfill
где $\omega_i$~-- массовая доля $i$-го компонента в смесевом потоке; $\omega_{i, j}$~-- массовая доля $i$-го компонента в $j$-ом потоке; $G_j$~-- массовый расход $j$-го потока; $j$~-- индекс потока; $i$~-- индекс компонента в системе; $n$~-- число потоков, подаваемых на смешение.
\vfill
\end{frame}

\contactsframe[\Large \textbf{Благодарю за внимание!}]{

\includegraphics[width=.05\textwidth]{pics/home} \quad Учебный корпус №2, ауд. 136 \\
\includegraphics[width=.05\textwidth]{pics/mail} \quad chuva@tpu.ru \\
\includegraphics[width=.03\textwidth]{pics/tel} \quad +7-962-782-66-15
}

\end{document}

