% !TeX document-id = {d14e3550-73bb-4219-9039-3bfa77854141}
% !BIB TS-program = biber

\documentclass[a4, 12pt, openany]{book}
\usepackage[left=2.5cm, right=2.0cm, top=2.0cm, bottom=2.0cm]{geometry}
\usepackage[T2A]{fontenc}
\usepackage[utf8]{inputenc}	
\usepackage[english, main=russian]{babel}
%\usepackage{polyglossia}
%\setdefaultlanguage{russian} %язык по умолчанию
%\setotherlanguage{english}
%\hyphenpenalty=10000

\usepackage{fontspec}
\setmainfont{Times New Roman}     % задаёт \rmfamily, основной шрифт документа
\setsansfont{Arial}     % задаёт \sffamily, шрифт без засечек
\setmonofont[Scale=0.9]{Consolas} % задаёт \ttfamily, моношрифт

\usepackage{import}
\usepackage{indentfirst}
\usepackage{textcase}
\usepackage{cmap}
\usepackage{graphicx}
\graphicspath{{pics/}}
\usepackage[citecolor=black]{hyperref}
\usepackage{pythonhighlight}
\usepackage{wrapfig}

\usepackage{tabularx}

%\usepackage[labelsep=endash]{caption}
\usepackage[nooneline]{caption}
\DeclareCaptionTextFormat{center}{\centering{#1}}
\DeclareCaptionLabelFormat{figure}{Рисунок~#2}
\captionsetup[table]{justification=raggedleft, 	
	labelsep=newline, 
	%	textfont=it, 
	textformat=center, 
	position=top, 
	skip=5pt
}
\captionsetup[figure]{justification=centering,
	labelsep=endash, 
	labelformat=figure
	%	textfont=it
}
\captionsetup[lstlisting]{justification=raggedright,
	labelsep=endash}

\usepackage{array,tabularx,tabulary,booktabs} % Дополнительная работа с таблицами
\usepackage{longtable}  % Длинные таблицы
\usepackage{multirow} % Слияние строк в таблице
\usepackage{amsmath, amsfonts, amssymb}

\usepackage{xcolor}
\definecolor{maroon}{cmyk}{0, 0.87, 0.68, 0.32}
\definecolor{halfgray}{gray}{0.55}
\definecolor{ipython_frame}{RGB}{207, 207, 207}
\definecolor{ipython_bg}{RGB}{247, 247, 247}
\definecolor{ipython_red}{RGB}{186, 33, 33}
\definecolor{ipython_green}{RGB}{0, 128, 0}
\definecolor{ipython_cyan}{RGB}{64, 128, 128}
\definecolor{ipython_purple}{RGB}{170, 34, 255}

\definecolor{linkcolor}{HTML}{0000FF} % цвет гиперссылок
\definecolor{urlcolor}{HTML}{800080} % цвет ссылок
\definecolor{backcolour}{rgb}{0.95,0.95,0.92}

\usepackage{type1ec}% Чтобы можно было делать буквицы высотой 3 строки и более.
%\usepackage{fix-cm}% Не уловил разницы.
\usepackage[T1,T2A]{fontenc}
%\usepackage[cp1251]{inputenc}

\usepackage{xecyr}
\usepackage{lettrine} % БУКВИЦЫ
\usepackage{yfonts}
\setcounter{DefaultLines}{2} % Высота буквицы в строках.
\renewcommand{\DefaultLoversize}{0.5} % по умолчанию 0, минимум 0.17
\renewcommand{\DefaultLhang}{0} % по умолчанию 0 — выступ буквицы на поле

%\renewcommand{\LettrineFontHook}{\fontseries{bx}} % Начертание буквицы, по умолчанию обычное.
\renewcommand{\LettrineTextFont}{\relax}% Начертание текста рядом с буквицей, по умолчанию капитель.

\hypersetup{pdfstartview=FitH,  linkcolor=linkcolor,urlcolor=urlcolor, colorlinks=true}

\usepackage{tocloft}
%\renewcommand{\cftchapfont}{\bfseries}
%\renewcommand{\cftchappagefont}{\bfseries}
\renewcommand{\cftchappresnum}{Глава }
%\renewcommand{\cftchapaftersnum}{:}
\renewcommand{\cftchapnumwidth}{4.5em}

\usepackage{listings}
\lstset{
	breaklines=true,
	%
	extendedchars=true,
	literate=
	{á}{{\'a}}1 {é}{{\'e}}1 {í}{{\'i}}1 {ó}{{\'o}}1 {ú}{{\'u}}1
	{Á}{{\'A}}1 {É}{{\'E}}1 {Í}{{\'I}}1 {Ó}{{\'O}}1 {Ú}{{\'U}}1
	{à}{{\`a}}1 {è}{{\`e}}1 {ì}{{\`i}}1 {ò}{{\`o}}1 {ù}{{\`u}}1
	{À}{{\`A}}1 {È}{{\'E}}1 {Ì}{{\`I}}1 {Ò}{{\`O}}1 {Ù}{{\`U}}1
	{ä}{{\"a}}1 {ë}{{\"e}}1 {ï}{{\"i}}1 {ö}{{\"o}}1 {ü}{{\"u}}1
	{Ä}{{\"A}}1 {Ë}{{\"E}}1 {Ï}{{\"I}}1 {Ö}{{\"O}}1 {Ü}{{\"U}}1
	{â}{{\^a}}1 {ê}{{\^e}}1 {î}{{\^i}}1 {ô}{{\^o}}1 {û}{{\^u}}1
	{Â}{{\^A}}1 {Ê}{{\^E}}1 {Î}{{\^I}}1 {Ô}{{\^O}}1 {Û}{{\^U}}1
	{œ}{{\oe}}1 {Œ}{{\OE}}1 {æ}{{\ae}}1 {Æ}{{\AE}}1 {ß}{{\ss}}1
	{ç}{{\c c}}1 {Ç}{{\c C}}1 {ø}{{\o}}1 {å}{{\r a}}1 {Å}{{\r A}}1
	{€}{{\EUR}}1 {£}{{\pounds}}1
}

%%
%% Python definition (c) 1998 Michael Weber
%% Additional definitions (2013) Alexis Dimitriadis
%% modified by me (should not have empty lines)
%%
\lstdefinelanguage{iPython}{
	morekeywords={access,and,break,class,continue,def,del,elif,else,except,exec,finally,for,from,global,if,import,in,is,lambda,not,or,pass,print,raise,return,try,while, nonlocal, yield, with},%
	%
	% Built-ins
	morekeywords=[2]{abs,all,any,basestring,bin,bool,bytearray,callable,chr,classmethod,cmp,compile,complex,delattr,dict,dir,divmod,enumerate,eval,execfile,file,filter,float,format,frozenset,getattr,globals,hasattr,hash,help,hex,id,input,int,isinstance,issubclass,iter,len,list,locals,long,map,max,memoryview,min,next,object,oct,open,ord,pow,property,range,raw_input,reduce,reload,repr,reversed,round,set,setattr,slice,sorted,staticmethod,str,sum,super,tuple,type,unichr,unicode,vars,xrange,zip,apply,buffer,coerce,intern, ascii, as, assert},%
	%
	sensitive=true,%
	morecomment=[l]\#,%
	morestring=[b]',%
	morestring=[b]",%
	%
	morestring=[s]{'''}{'''},% used for documentation text (mulitiline strings)
	morestring=[s]{"""}{"""},% added by Philipp Matthias Hahn
	%
	morestring=[s]{r'}{'},% `raw' strings
	morestring=[s]{r"}{"},%
	morestring=[s]{r'''}{'''},%
	morestring=[s]{r"""}{"""},%
	morestring=[s]{u'}{'},% unicode strings
	morestring=[s]{u"}{"},%
	morestring=[s]{u'''}{'''},%
	morestring=[s]{u"""}{"""},%
	morestring=[s]{b'}{'},% byte strings
	morestring=[s]{b"}{"},%
	morestring=[s]{b'''}{'''},%
	morestring=[s]{b"""}{"""},%
	%
	% {replace}{replacement}{lenght of replace}
	% *{-}{-}{1} will not replace in comments and so on
	literate=
	{á}{{\'a}}1 {é}{{\'e}}1 {í}{{\'i}}1 {ó}{{\'o}}1 {ú}{{\'u}}1
	{Á}{{\'A}}1 {É}{{\'E}}1 {Í}{{\'I}}1 {Ó}{{\'O}}1 {Ú}{{\'U}}1
	{à}{{\`a}}1 {è}{{\`e}}1 {ì}{{\`i}}1 {ò}{{\`o}}1 {ù}{{\`u}}1
	{À}{{\`A}}1 {È}{{\'E}}1 {Ì}{{\`I}}1 {Ò}{{\`O}}1 {Ù}{{\`U}}1
	{ä}{{\"a}}1 {ë}{{\"e}}1 {ï}{{\"i}}1 {ö}{{\"o}}1 {ü}{{\"u}}1
	{Ä}{{\"A}}1 {Ë}{{\"E}}1 {Ï}{{\"I}}1 {Ö}{{\"O}}1 {Ü}{{\"U}}1
	{â}{{\^a}}1 {ê}{{\^e}}1 {î}{{\^i}}1 {ô}{{\^o}}1 {û}{{\^u}}1
	{Â}{{\^A}}1 {Ê}{{\^E}}1 {Î}{{\^I}}1 {Ô}{{\^O}}1 {Û}{{\^U}}1
	{œ}{{\oe}}1 {Œ}{{\OE}}1 {æ}{{\ae}}1 {Æ}{{\AE}}1 {ß}{{\ss}}1
	{ç}{{\c c}}1 {Ç}{{\c C}}1 {ø}{{\o}}1 {å}{{\r a}}1 {Å}{{\r A}}1
	{€}{{\EUR}}1 {£}{{\pounds}}1,
	%
	literate=
	*{+}{{{\color{ipython_purple}+}}}1
	{-}{{{\color{ipython_purple}-}}}1
	{*}{{{\color{ipython_purple}$^\ast$}}}1
	{/}{{{\color{ipython_purple}/}}}1
	{^}{{{\color{ipython_purple}\^{}}}}1
	{?}{{{\color{ipython_purple}?}}}1
	{!}{{{\color{ipython_purple}!}}}1
	{\%}{{{\color{ipython_purple}\%}}}1
	{<}{{{\color{ipython_purple}<}}}1
	{>}{{{\color{ipython_purple}>}}}1
	{|}{{{\color{ipython_purple}|}}}1
	{\&}{{{\color{ipython_purple}\&}}}1
	{~}{{{\color{ipython_purple}~}}}1
	%
	%	{==}{{{\color{ipython_purple}==}}}2
	%	{<=}{{{\color{ipython_purple}<=}}}2
	%	{>=}{{{\color{ipython_purple}>=}}}2
	%	
	%	{+=}{{{\color{ipython_purple}>=}}}2
	%	{&=}{{{\color{ipython_purple}>=}}}2
	%	{-=}{{{\color{ipython_purple}>=}}}2
	%	{|=}{{{\color{ipython_purple}>=}}}2
	%	
	%	{*=}{{{\color{ipython_purple}>=}}}2
	%	{^=}{{{\color{ipython_purple}>=}}}2
	%	{/=}{{{\color{ipython_purple}>=}}}2
	%	{>>=}{{{\color{ipython_purple}>=}}}2
	%	
	%	{\%=}{{{\color{ipython_purple}>=}}}2
	%	{<<=}{{{\color{ipython_purple}>=}}}2
	%	{**=}{{{\color{ipython_purple}>=}}}2
	%	{//=}{{{\color{ipython_purple}>=}}}2
	%
	{+=}{{{+=}}}2
	{-=}{{{-=}}}2
	{*=}{{{$^\ast$=}}}2
	{/=}{{{/=}}}2,
	%
	%	identifierstyle=\color{red}\ttfamily,
	commentstyle=\fontsize{10pt}{10}\color{ipython_cyan}\ttfamily,
	texcl=true,
	keepspaces=true,
	stringstyle=\fontsize{10pt}{10}\color{ipython_red}\ttfamily,
	%	keepspaces=true,
	showspaces=false,
	showstringspaces=false,
	%
	rulecolor=\color{ipython_frame},
	frame=leftline,
	%	frameround=ffff,
	framexleftmargin=5mm,
	columns=fullflexible
	numbers=left,
	numberstyle=\tiny\color{halfgray},
	numbersep=5pt,
	%
	%
	%	backgroundcolor=\color{ipython_bg},
	%   extendedchars=true,
	basicstyle=\fontsize{10pt}{10}\ttfamily,
	keywordstyle=\fontsize{10pt}{10}\color{ipython_green}\ttfamily,
	escapechar=\¢,escapebegin=\color{ipython_red},
}


\usepackage{icomma}
%\DeclareUnicodeCharacter{00A0}{ }

%\usepackage[Conny]{fncychap}
\usepackage{fancyhdr}
\pagestyle{fancy}
\setlength{\headheight}{15.0pt}
\fancyhf{}
\fancyhead[LE,RO]{\bfseries\thepage}
%\fancyhead[RE]{\bfseries\nouppercase{\rightmark}}
%\fancyhead[LO]{\bfseries\nouppercase{\leftmark}}
\renewcommand{\headrulewidth}{0.5pt}
\renewcommand{\footrulewidth}{0pt}

\usepackage{pgf}
\usepackage{capt-of}
\usepackage{amsmath}
\usepackage{nicefrac}

\usepackage[backend=biber, 
style=gost-numeric,  
autolang=other, 
clearlang=true, 
bibencoding=utf8,
sorting=none
]{biblatex}
\renewcommand*{\multicitedelim}{\addcomma\space} 
\usepackage{csquotes}
\addbibresource{ref.bib}

\begin{document}
	
	\begin{titlepage}
	\begin{center}
\sffamily	\small{\MakeUppercase {Министерство науки и высшего образования
			Российской Федерации}}\\
		\footnotesize {Федеральное государственное автономное образовательное учреждение  высшего образования}\\
	\small	\MakeTextUppercase{\textbf{<<Национальный исследовательский} \\ \textbf{Томский политехнический университет>>}} \\
		\hrulefill
		\\[4cm]


		\bigskip

		{
			\Large \MakeUppercase{\textbf{Программный код}} \\

			Системный анализ процессов химической технологии \\
			Лабораторная работа №8 \\
			<<Идентификация кинетических параметров>>
			\
			 \bigskip



		}

\vspace{0.1\textheight}
\begin{minipage}{0.46\linewidth}
	\hfil
\end{minipage}



		\vfill
		Томск~--~2023~г.
	\end{center}

\end{titlepage}
	
	%\addcontentsline{toc}{chapter}{Начало работы}
	
%	\include{secondpage}
	\renewcommand\contentsname{Содержание}
%	\tableofcontents
	
	\lstlistoflistings
	%\part{Программирование на языке Python}


%\chapter{Архитектура программного решения}\label{ch:архитектура}
%%\addcontentsline{toc}{chapter}{Расчет физико-химических свойств}


\lstinputlisting[language=iPython, caption={Модуль констант \texttt{constants.py}}, numbers=none]{../scripts/constants.py}

\newpage
\lstinputlisting[language=iPython, caption={Вспомогательные функции (\texttt{converters\_and\_functions.py})}, numbers=none]{../scripts/converters\_and\_functions.py}

\newpage
\lstinputlisting[language=iPython, caption={Описание класса \texttt{Flow} (\texttt{flows.py})}, numbers=none]{../scripts/flows.py}

\newpage
\lstinputlisting[language=iPython, caption={Описание класса \texttt{Mixer} (\texttt{mixer.py})}, numbers=none]{../scripts/mixer.py}

\newpage
\lstinputlisting[language=iPython, caption={Описание кинетической схемы (\texttt{kinetic.py})}, numbers=none]{../scripts/kinetic.py}

\newpage
\lstinputlisting[language=iPython, caption={Описание класса \texttt{Reactor} (\texttt{reactors.py})}, numbers=none]{../scripts/reactors.py}

\newpage
\lstinputlisting[language=iPython, caption={Описание основного модуля (\texttt{main.py})}, numbers=none]{../scripts/main.py} % Введение
	\include{chapter2} % Как пользователь выполняет программы
	\include{chapter3}
%	\include{chapter20} % IPython
%	\include{chapter5} % Числовые типы в Python
%	\include{chapter6} % Динамическая типизация
%	\include{chapter7} % Строки в Python
%	\include{chapter8} % Списки и словари
%	\include{chapter9} % Кортежи, файлы и прочие типы 
%	\include{chapter10} % Операторы и синтаксис
%	\include{chapter11} % Проверки if и правила синтаксиса
%	\include{chapter12} % Циклы while и for
%	\include{chapter13} % Итерации и итерируемые объекты
%	\include{chapter14} % Функции
%	\include{chapter15} % Области видимости
%	\include{chapter16} % Расширенные возможности функций
%	\include{chapter17} % Генераторы и генераторные выражения
%	\include{chapter18} % Модули
%	\include{chapter19} % Классы и объектно-ориентрированное программирование
%	% Часть II
%%	\include{intro} % Веедение
%	\include{chapter21} % NumPy
%	\include{chapter23} % Matplotlib
%	\include{chapter24} % SciPy
%	\include{chapter25} % Численные методы решения нелинейных уравнений
%	\include{chapter30} % Методы решения систем линейных уравнений
%	\include{chapter26} % Обработка экспериментальных данных
%	\include{chapter27} % Численные методы интегрирования
%	\include{chapter28} % Численные методы решения ОДУ
%	\include{chapter29} % Сисемы дифференциальных уравнений
%	\include{chapter31} % Методы оптимизации
%	\include{conclusion}
%	
%	\lstlistoflistings
%	
%	\addcontentsline{toc}{chapter}{Листинги}
%	
%	\printbibliography
%	
	\addcontentsline{toc}{chapter}{Список литературы}
%	
%	\newpage
%	
%	\include{lastpage}
	
\end{document}
